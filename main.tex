\documentclass{article}
\usepackage{indentfirst} % indent first paragraph
\usepackage[style=alphabetic,sorting=ynt]{biblatex}
\usepackage{amsmath,amssymb,amsfonts,amsthm}
\usepackage{xcolor}
\usepackage[colorlinks]{hyperref}
\usepackage[shortlabels]{enumitem}
\usepackage[margin=0.75in]{geometry}
\usepackage{mathpartir}
\usepackage{scalerel}

\RequirePackage[T1]{fontenc}
\RequirePackage[tt=false, type1=true]{libertine}
\RequirePackage[varqu]{zi4}
\RequirePackage[libertine]{newtxmath}

\addbibresource{refs.bib}
\addbibresource{short.bib}
\usepackage{xcolor}
\usepackage{amsmath}
\usepackage{amssymb}
\usepackage{multirow,longtable,booktabs,array}
\usepackage{amsfonts}
\usepackage{listings}
\usepackage{fontspec}
\usepackage{graphicx}
\usepackage{hyperref}
\usepackage{float}
\usepackage{tikz}
\usepackage{fancybox} % https://tex.stackexchange.com/a/85492
\usepackage{mathpartir}

\usepackage[style=alphabetic,sorting=ynt,backend=bibtex]{biblatex}
\addbibresource{refs.bib}


\newcommand\rarr\rightarrow
\newcommand\gray{\textcolor{gray}}

\setmonofont{iosevka-custom-regular.ttf}
\setmainfont{Roboto-Regular.ttf}

\hypersetup{
    colorlinks=true,
    linkcolor=blue,
    urlcolor=cyan,
}


\newcommand{\Lift}{{\Uparrow}}
\newcommand{\spl}{{\sim}}
\newcommand{\qut}[1]{\langle #1\rangle}

\renewcommand{\mit}[1]{\mathit{#1}}
\newcommand{\msf}[1]{\mathsf{#1}}
\newcommand{\mbb}[1]{\mathbb{#1}}
\newcommand{\mbf}[1]{\mathbf{#1}}
\newcommand{\bs}[1]{\boldsymbol{#1}}
\newcommand{\wh}[1]{\widehat{#1}}
\newcommand{\ext}{\triangleright}
\newcommand{\Code}{\msf{Code}}
\newcommand{\El}{\msf{El}}
\newcommand{\lam}{\msf{lam}}
\newcommand{\app}{\msf{app}}
\newcommand{\NatElim}{\msf{NatElim}}
\newcommand{\y}{\msf{y}}

\newcommand{\U}{\msf{U}}
\newcommand{\Con}{\msf{Con}}
\newcommand{\Sub}{\msf{Sub}}
\newcommand{\Ty}{\msf{Ty}}
\newcommand{\Tm}{\msf{Tm}}

\newcommand{\Set}{\mathsf{Set}}
\newcommand{\Prop}{\mathsf{Prop}}
\newcommand{\Rep}{\msf{Rep}}
\newcommand{\blank}{{\mathord{\hspace{1pt}\text{--}\hspace{1pt}}}}
\newcommand{\emb}[1]{\ulcorner#1\urcorner}


\newcommand{\refl}{\msf{refl}}
\newcommand{\Bool}{\msf{Bool}}
\newcommand{\true}{\msf{true}}
\newcommand{\false}{\msf{false}}
\newcommand{\True}{\msf{True}}
\newcommand{\False}{\msf{False}}
\newcommand{\List}{\msf{List}}
\newcommand{\nil}{\msf{nil}}
\newcommand{\cons}{\msf{cons}}
\newcommand{\Nat}{\msf{Nat}}
\newcommand{\zero}{\msf{zero}}
\newcommand{\suc}{\msf{suc}}
\renewcommand{\tt}{\msf{tt}}
\newcommand{\fst}{\msf{fst}}
\newcommand{\snd}{\msf{snd}}
\newcommand{\mylet}{\msf{let}}
\newcommand{\emptycon}{\scaleobj{.75}\bullet}
\newcommand{\id}{\msf{id}}

\newcommand{\p}{\mathsf{p}}
\newcommand{\q}{\mathsf{q}}


\theoremstyle{plain}
% \newtheorem{theorem}{Theorem}[section]

\theoremstyle{definition}
% \newtheorem{definition}{Definition}[section]
% \newtheorem{example}{Example}[subsection]
\newtheorem{property}{Property}[section]
% \newtheorem{corollary}{Corollary}[theorem]
% \newtheorem{lemma}[theorem]{Lemma}

\theoremstyle{remark}
\newtheorem{notation}{Notation}
\newtheorem{remark}{Remark}



\title{Introduction to Staged Compilation and Two-Level Type Theory}
\date{}
\author{Yulong Liu\and Youzhang Sun}
\begin{document}

\maketitle

\section{Introduction to Staged Compilation}
The purpose of staged compilation is to write expressive metaprograms that generate code with the guarantees that the generated code is well-formed. To justify the well-formedness of the code output, the model of two-level type theory (2LTT) \cite{2ltt} is employed as a formal typing system for staged compilation. While languages such as MetaML \cite{10.1145/258994.259019} supports metaprogramming, 2LTT additionally supports dependent types. 
% Therefore, 2LTT can ensure that each metaprogram generates a well-formed output code and provide the additional expressiveness from dependent types.
Therefore, 2LTT contributes to the field of programming language and type theory by introducing dependent types to metaprogramming.

% Staged compilation is the process of compiling a program from one language to another through a staging algorithm. For a compiled language, the compiler translates the source code written in the compile time language to a program written in the runtime language, often as the language of CPU instructions. In languages that support code generation, there also exist two stages between programs with code-generating annotations (e.g. macros, generics, function inlining) to an output code without these annotations (e.g. by substituting the inlined function inside the body of the caller function). In the case of annotations as additional syntax for the compile time language, the compiler performs staged compilation by staging these annotations away; in other words, it substitutes these annotations with code in the runtime language.

% Examples of code generation with staged compilation include macros in C and generics in Rust. The preprocessor of C takes the source code and substitutes each usage of macros with their corresponding replacement code, thus performing code generation. The Rust compiler behaves similarly with respect to programs that uses generic functions: it makes copies of each generic function with the type parameters substituted by concrete types.

In this paper, we focus on metaprogramming with two stages. We index these stages as stage $0$ and stage $1$. Each stage has a language which we will further formalise into a type system. We use the term \emph{staging algorithm} to refer the process of transforming (i.e. staging) a metaprogram to a program that only uses the stage $0$ language. A metaprogram is a term with stage $0$ type but uses type/terms from the stage $1$ language through staging annotations. To explain how a metaprogram can use stage $1$ language despite being at stage $0$, we first describe the interaction between these two stages.



\subsection{Interaction between Stages}
While 2LTT provides safety about the output code staged from a metaprogram, the only way to interact between stages is through these three staging operations: lifting, quoting, and splicing. 

\begin{itemize}
    \item If we have a stage 0 type $A_0$, we can \emph{lift} $A_0$, denoted as $\Lift A_0$. The type $\Lift A_0$ describes the type of metaprograms that compute a stage $0$ expression of type $A_0$. 
    \item If we have a stage 0 term $t$ of type $A_0$, we can \emph{quote} $t$, denoted as $\qut{t}$. The term $\langle t\rangle$ describes a metaprogram that immediately yields $t$.
    \item If we have a stage 1 term $s$ of type $\Lift A_0$, we can \emph{splice} $s$, denoted as $\spl s$. The term $\spl s$ describes a term that is to be executed during staging with the resulting expression substituted back to the output code. 


    % \item If we lift a stage $0$ type $A_0$, denoted as ${\Uparrow}A_0$, the lifted type becomes a stage $1$ type. 
    % \item We can also lift a stage $0$ term $x_0:A_0$ to stage $1$ through quoting, denoted as $\langle x_0\rangle:{\Uparrow}A_0$. The notation $\langle t\rangle$ describes a metaprogram that immediately yields $t$.
    % \item The splice operation brings a quoted term back to stage $0$. In other words, if $t_0$ is a ${\Uparrow}A_0$, then the splice of the term, denoted as ${\sim}t_0:A_0$, means that $t_0$   
\end{itemize}

The combinations of these staging operations allows us to lift stage $0$ terms and types to stage $1$, and bring lifted terms back to stage $0$.



\subsection{Example of A Staged Program}\label{example-staged}

In this section, we start with an example of a program without staging annotations; then we will convert this program into a metaprogram to demonstrate the usage of staging operations. After introducing inference rules of 2LTT in later sections, we will revisit this example by type-checking and staging it according to the inference rules. 

For now, let's consider a programming language that provides addition for natural numbers. We can implement a multiplication algorithm using the provided addition function recursively. Thus, we can write a program, called $\mathsf{double}:\mathsf{Nat}\to\mathsf{Nat}$ that fixes the first argument of $\mathsf{mul}$:

$$
\begin{aligned}
    % \mathsf{Nat}&:\mathsf{U}\\
    \mathsf{zero}&:\mathsf{Nat}\\
    \mathsf{suc}&:\mathsf{Nat}\to\mathsf{Nat}\\
    % \mathsf{NatElim}&:(P:\mathsf{Nat}\to\mathsf{U})\to((i:\mathsf{Nat})\to P\,i\to P\,(\mathsf{suc}\,n))\to P\,\mathsf{zero}\to(n:\mathsf{Nat})\to P\,n\\
    % \mathsf{NatElim}&\,P\,s\,z\,\mathsf{zero}:=\mathsf{zero}\\
    % \mathsf{NatElim}&\,P\,s\,z\,(\mathsf{suc}\,n):=s\,n\,(\mathsf{NatElim}\,P\,s\,z\,n)\\
    % \mathsf{iter}&:(A:\mathsf{U})\to\mathsf{Nat}\to A\to (A\to A)\to A\\
    % \mathsf{iter}&\,A\,\mathsf{zero}\,z\,s:=s\,z\\
    % \mathsf{iter}&\,A\,(\mathsf{suc}\,n)\,z\,s:=s\,(\mathsf{iter}\,A\,n\,z\,s)\\
    % \\
    \mathsf{add}&:\mathsf{Nat}\to\mathsf{Nat}\to\mathsf{Nat}\\
    % \mathsf{add}&\,a\,b:=\mathsf{iter}\,a\,b\,\mathsf{suc}\\
    % \mathsf{add}&:=\lambda\,a\,b.\,\mathsf{iter}\,a\,b\,\mathsf{suc}\\
    \\
    \mathsf{mul}&:\mathsf{Nat}\to\mathsf{Nat}\to\mathsf{Nat}\\
    % \mathsf{mul}&:=\lambda\,a\,b.\,\mathsf{iter}\,a\,\mathsf{zero}\,(\mathsf{add}\,b)\\
    \mathsf{mul}&\,\mathsf{zero}\,x=\mathsf{zero}\\
    \mathsf{mul}&\,(\mathsf{suc}\,n)\,x=\mathsf{add}\,x\,(\mathsf{mul}\,n\,x)\\
    \\
    \mathsf{double}&:\mathsf{Nat}\to\mathsf{Nat}\\
    \mathsf{double}&:=\lambda\,x.\,\mathsf{mul}\,(\mathsf{suc}\,(\mathsf{suc}\,\mathsf{zero}))\,x\\
    % \mathsf{triple}&:\mathsf{Nat}\to\mathsf{Nat}\\
    % \mathsf{triple}&:=\lambda\,x.\,\mathsf{mul}\,3\,x\\
\end{aligned}
$$

Currently, there is no distinction between the stage of our function $\mathsf{mul}$ and $\mathsf{add}$. For the purpose of demonstrating the effect of staging a metaprogram, let us lift the function $\mathsf{mul}$ to stage $1$ and keep $\mathsf{add}$ at stage $0$. The staging algorithm will stage $\mathsf{double}$ into a program with no occurrences of $\mathsf{mul}$. In other words, we can stage the code $\lambda\,x.\,\mathsf{mul}\,(\mathsf{suc}\,(\mathsf{suc}\,\mathsf{zero}))\,x$ into $\lambda\,x.\,\mathsf{add}\,x\,(\mathsf{add}\,x\,\mathsf{zero})$ through substitution with its definition during staging. 

Therefore, we rewrite these function as $\mathsf{mul}_1$ and $\mathsf{add}_0$ to indicate their stage. Because both functions $\mathsf{add}_0$ and $\mathsf{mul}_1$ take a natural number as argument but compute in different stages, we also need each stage to have a type for describing naturals. Thus to distinguish between naturals in these two stages, we index the type by its stage. We define $\mathsf{Nat}_0$ to be the type of natural numbers at stage $0$, and $\mathsf{Nat}_1$ to be the type of natural numbers at stage $1$. We treat these two types as separate types, thus the constructors of these naturals are also indexed: $\mathsf{zero}_1$ and $\mathsf{suc}_1$ for stage $1$, and $\mathsf{zero}_0$ and $\mathsf{suc}_0$ for stage $0$. 

$$
\begin{aligned}
    \mathsf{zero}_0&:\mathsf{Nat}_0\\
    \mathsf{suc}_0&:\mathsf{Nat}_0\to\mathsf{Nat}_0\\
    \mathsf{add}_0&:\mathsf{Nat}_0\to\mathsf{Nat}_0\to\mathsf{Nat}_0\\
    \\
    \mathsf{zero}_1&:\mathsf{Nat}_1\\
    \mathsf{suc}_1&:\mathsf{Nat}_1\to\mathsf{Nat}_1\\
    \\
    \mathsf{mul}_1&:\mathsf{Nat}_1\to{\Uparrow}\mathsf{Nat}_0\to{\Uparrow}\mathsf{Nat}_0\\
    \mathsf{mul}_1&\,\mathsf{zero}_1\,t=\langle \mathsf{zero}_0\rangle\\
    \mathsf{mul}_1&\,(\mathsf{suc}_1\,n)\,t=\langle \mathsf{add}_0\,{\sim}t\,{\sim}(\mathsf{mul}_1\,n\,t)\rangle\\
    \\
    \mathsf{double}_0&:\mathsf{Nat}_0\to\mathsf{Nat}_0\\
    \mathsf{double}_0&:=\lambda\,x.\,{\sim}(\mathsf{mul}_1\,(\mathsf{suc}_1\,(\mathsf{suc}_1\,\mathsf{zero}_1))\,\langle x\rangle)\\
\end{aligned}
$$

We also rewrite our original double program as $\mathsf{double}_0:\mathsf{Nat}_0\to\mathsf{Nat}_0$ since it is the program we want to stage into the output code. However, the argument of the function $\mathsf{double}_0$, namely $x:\mathsf{Nat}_0$, is passed to $\mathsf{mul}_1$, a function at stage $1$. Therefore, we need to quote this argument as $\langle x\rangle:{\Uparrow}\mathsf{Nat}_0$ to lift $x$ to stage $1$ for the metaprogram $\mathsf{mul}_1$ to compute.

% Notice that the arguments of $\mathsf{mul}_1:\Nat_1\to\Lift\Nat_0\to\Lift\Nat_0$ have different types. The reason is that, while the we fix the first argument of $\mathsf{mul}_1$ in double as $\suc_1\,(\suc_1\,\mathsf{zero}_1)$, the second argument is passed as the quoted argument of $\mathsf{double}_0:\Nat_0\to\Nat_0$. That is, for the arguments of $\mathsf{mul}_1$, we compute the first argument during staging (thus it's a stage $1$ term) while the second argument comes from the argument of $\mathsf{double}_0$ (thus a quoted stage $0$ term).

Therefore, our program $\mathsf{double}_0$ takes a $x:\mathsf{Nat}_0$ as a stage $0$ term, quotes it as $\langle x\rangle:{\Uparrow}A$ to pass it to the metaprogram $\mathsf{mul}_1\,(\mathsf{suc}_1\,(\mathsf{suc}_1\,\mathsf{zero}_1))$, which would gives us a lifted term in stage $1$. Then, we can splice the output of $\mathsf{mul}_1$ to get back a stage $0$ term. 

For the metaprogram $\mathsf{mul}_1$, we follow its recursive definition with respect to the first argument, a $\mathsf{Nat}_1$ from stage $1$. Since $\mathsf{mul}_1$ is a metaprogram that generates a lifted code for stage $0$, namely with type ${\Uparrow}\mathsf{Nat}_0$, we splice the output inside $\mathsf{double}_0$ to bring the term back to stage $0$. Therefore, we quote the base case as $\langle \mathsf{zero}_0\rangle$ and the recursive step as $\langle \mathsf{add}_0\,\cdots\rangle$ to yield these terms during staging. In the recursive step, as $t$ is passed as a ${\Uparrow}\mathsf{Nat}_0$, namely $\langle x\rangle$ from $\mathsf{double}_0$, we splice it to get $x:\mathsf{Nat}_0$ and pass it to the stage $0$ function $\mathsf{add}_0$.

Once we feed this program to the staging algorithm, we will get a well-typed program where the occurrence of $\mathsf{mul}_1$ inside $\mathsf{double}_0$ is substituted by its compile time definition during staging:

$$
\begin{aligned}
    \mathsf{zero}_0&:\mathsf{Nat}_0\\
    \mathsf{suc}_0&:\mathsf{Nat}_0\to\mathsf{Nat}_0\\
    \mathsf{add}_0&:\mathsf{Nat}_0\to\mathsf{Nat}_0\to\mathsf{Nat}_0\\
    \\
    \mathsf{double}_0&:\mathsf{Nat}_0\to\mathsf{Nat}_0\\
    \mathsf{double}_0&:=\lambda\,x.\,\mathsf{add}_0\,x\,(\mathsf{add}_0\,x\,\mathsf{zero}_0)\\
\end{aligned}
$$




\subsection{Soundness of Staging}
% While we introduced explicit notations for staging a metaprogram, it is paramount to justify the soundness of these notations and the staging algorithm itself as well. The type system of 2LTT, by restricting the interaction between stage $1$ and stage $0$, guarantees that our staged output is necessarily well-typed and well-staged. That is, all the special operations for moving between stages are guaranteed to be evaluated and substituted by stage $0$ code. As a result, for the purpose of type checking our metaprograms to justify the soundness of staging, we introduce two-level type theory.

% --- Alternative ---

The result of staging $\mathsf{double}_0$ was a well-typed function that only uses terms and constructors native to stage 0. We would like to generalize so any metaprogram should be staged to a program that only uses terms and constructors from stage 0. We use the term ``sound'' to describe such staging algorithm.

\begin{definition}[Soundness, Well-staged, Well-formedness]
    A staging algorithm is \emph{sound} if for all input program with stage 0 type, the algorithm outputs a well-typed program containing only stage 0 terms / constructors, and no lift, quote, or splice.
    
    A well-typed output program that contains only stage 0 terms / constructors, and no lift, quote, or splice, is called \emph{well-staged} or \emph{well-formed}.
\end{definition} 
To justify the soundness of staging algorithm, we use two-level type theory, which the remaining paper will explore.




\section{Introduction to Two-Level Type Theory}

Two-level type theory (2LTT), as the name suggests, extends type theory to two levels, which in our case, are the stages $0$ and $1$. 2LTT is useful since we can extend the language of stage $0$ as the stage $1$ language to derive useful properties that are not expressable within stage $0$; namely, the soundness of a staging algorithm.

To apply 2LTT in staged programming, we consider the languages used in the two stages as separate type systems. This separation provides support for a wide range of languages even with completely different syntax. For instance, staged compilation with 2LTT is applicable to domain specific languages whose implementations are in different languages (e.g. staging LINQ expressions to C\# method calls \cite{linq}). % 2LTT also restricts the interaction between these two theories through the three special operations: quoting, lifting, and splicing.

Since 2LTT formalises our metaprograms and output code into type theories, we provide an example of a 2LTT model that consists of universe hierarchies, type formers with dependent types, and formers/eliminators for natural numbers. We will also present a portion of the inference rules for the purpose of formalising the staging operations and working with natural numbers. Next, we will revisit the $\mathsf{double}_0$ metaprogram for type-checking and applying the staging algorithm with these inference rules. We will then round off this paper with some discussions and conclusion.



\subsection{Universes and Type Formers}

The types that we construct in both stages can be treated as terms as well; a type whose terms are also types is called a universe. As universes themselves are types, a hierarchy of universes is formed for each stage. We denote these universes as $\mathsf{U}_{i,j}$ for $i\in\{0,1\}$ and $j\in\mathbb N$ where $i$ indicates the stage and $j$ for the universe level, that is $\mathsf{U}_{i,j}$ is a term of $\mathsf{U}_{i,j+1}$. % The interpretation of types (including universes) as terms can be made explicit with the operators $\mathsf{El}$ and $\mathsf{Code}$, which we will cover in later sections in detail.

In staged programming, the universes $\mathsf{U}_{0,j}$ are inhabited by types from stage $0$ and $\mathsf{U}_{1,j}$ by types from stage $1$. For instance, $\mathsf{Nat}_0$ as a term has the type $\mathsf{U}_{0,0}$ whereas $\mathsf{Nat}_1$ as a term has the type $\mathsf{U}_{1,0}$. As mentioned before, we can lift a stage $0$ type to stage $1$, thus for any $A:\mathsf{U}_{0,j}$, we have ${\Uparrow}A:\mathsf{U}_{1,j}$.

To construct new types, there are type formers (also known as type constructors) for each universe and stage. Type formers takes types as arguments and construct a new type in the same stage. For instance, the $\Pi$-type constructor combines a domain type $A:\mathsf{U}_{i,j}$ and a codomain type $B:\mathsf{U}_{i,j}$ to form the function type $\Pi\,A\,B:\mathsf{U}_{i,j}$ which describes the type of a function that maps an $A$ to a $B$. We notice here $A$ and $B$ are from the same stage. All type formers require the input types to be from the same stage, and outputs a type of the same stage.
% This function type can also be generalised into a dependent function type $\Pi A B$, which we will formalise in 4.2 with inference rules.  
 


Up until now, we have given a informal description of the natural number type and staging operations to introduce staged compilation through examples. In the next sections, we formalise these ideas through inference rules.



\section{Inference Rules of 2LTT}

In this section, we analyze the inference rules of 2LTT. As 2LTT is an extension of Martin-Löf type theory, many concepts involve to dependent functions. Therefore, to denote the type of a dependent function, we use the notation $(x:A)\to B$ where the term $x$ may occur in the type $B$. We also use the alternative notation $(x:A)\to B\,x$, which clarifies that $B$ is a type dependent on $x$.

 
\subsection{Judgments}
We start by listing the forms of judgments defined for 2LTT relevant to our use. We will then explain the meaning of each judgment presented. 

Because there are two different stages in 2LTT, a type in one stage is often not available in the other stage. As 2LTT supports the hierarchy of universes, we need to take into consideration the level we are at when making a judgment. We assume index $i \in \{0, 1\}$ to indicate the stage, and $j \in \mathbb N$ for the universe level. 
\begin{notation}
    ``$\Gamma \vdash_{i, j} ...$'' is the notation for making the judgment in stage $i$ and universe level $j$. 
\end{notation}
We also use the following convention:
\begin{itemize}
    \item Uppercase Greek letters $\Gamma, \Delta$ for context
    \item Lowercase Greek letters $\sigma, \delta$ for context substitution
    \item Uppercase alphabet $A, B, C$ for types
    \item Lowercase alphabet $t, u, v$ for terms
\end{itemize}

\begin{definition}[Judgments of 2LTT]
    \begin{alignat*}{2}
      & \Gamma\vdash\hspace{6em}           && \text{\emph{context formation}}\\
     & \Gamma\vdash \sigma : \Delta       && \text{\emph{explicit substitution formation, assuming $\Gamma \vdash$ and $\Delta \vdash$}}\\
     % & \Gamma \vdash \sigma = \delta : \Delta && \text{\emph{substitution equality, assuming $\Gamma \vdash \sigma : \Delta$ and $\Gamma \vdash \delta : \Delta$}}\\
      & \Gamma\vdash_{i,j} A                && \text{\emph{type formation, assuming $\Gamma\vdash$}}\\
      & \Gamma\vdash_{i,j} t : A            && \text{\emph{term formation, assuming $\Gamma \vdash_{i,j} A$}}\\
      & \Gamma \vdash_{i,j} A = B           && \text{\emph{type equality, assuming $\Gamma \vdash_{i,j} A$ and $\Gamma \vdash_{i,j} B$}}\\
      & \Gamma \vdash_{i,j} t = u : A       && \text{\emph{term equality, assuming $\Gamma \vdash_{i,j} t : A$ and $\Gamma \vdash_{i,j} u : A$}}
    \end{alignat*}
\end{definition}

We now explain each judgment in detail.

\begin{definition}[Form of Judgment Context Formation]
    \begin{alignat*}{2}
        &\Gamma \vdash \hspace{0.5em} && \text{means ``$\Gamma$ is a context''}
    \end{alignat*}
We always have the empty context, denoted as $\emptycon$. Thus, the judgment ``$\emptycon \vdash$'' would be a believable judgment.
\end{definition} 


\begin{definition}[Substitution and De Bruijn Indices]
    \begin{alignat*}{2}
        &\Gamma \vdash\sigma:\Delta \hspace{0.5em} && \text{means ``under the context $\Gamma$, $\sigma$ is a substitution from $\Delta$''}
    \end{alignat*}

In short, a substitution is a mapping of terms and types from one context to another. In particular, $\Gamma\vdash\sigma:\Delta$ means $\sigma$ maps terms and types from the context $\Delta$ to the context $\Gamma$. If $\Delta\vdash_{i,j}A$, that is, $A$ is a type under the context $\Delta$, then we can apply the substitution to derive that $\Gamma\vdash_{i,j}A[\sigma]$, that is, $A[\sigma]$ is a type under the context $\Gamma$. We will discuss applying substitions in detail through inference rules in later sections.

For this paper, we only work with the \emph{weakening substitution} $\p$ for the purpose of employing De Bruijn indices. The De Bruijn index is a convenient way of refering to the terms in a context without putting them in it explicitly. Again, we will provide examples in the section of inference rules.
\end{definition}


\begin{definition}[Forms of Judgment for Type and Term Formation]
    \begin{alignat*}{2}
        & \Gamma \vdash_{i, j} A && \text{means ``$A$ is a type of stage $i$ , universe level $j$''} \\
        &\Gamma \vdash_{i, j} t:A \hspace{2em} && \text{means ``$t$ is a term of type  $A$ in stage $i$, universe level $j$ ''}
    \end{alignat*}
    
\end{definition}

\begin{example}
  The expression ``$\Gamma \vdash_{1, 10} A$'' means ``$A$ is a type in stage 1 and universe level 10''     
\end{example}


\begin{example}
    Consider the two $\Nat$ types for the two stages: $\Nat_0$ and $\Nat_1$. While the type $\Nat_0$ belong strictly in stage 0, the type $\Nat_1$ is in stage 1. 
    So $\Gamma \vdash_{0, j} \Nat_0$ would be a believable judgment, but $\Gamma \vdash_{1, j} \Nat_0$ is not believable.
\end{example}

The judgment ``$\Gamma \vdash_{i, j} A$'' makes the presupposition that $\Gamma$ is a valid context ($\Gamma \vdash$). The judgment ``$\Gamma \vdash_{i,j} t:A$'' assumes $A$ is a type in stage $i$ universe level $j$ ($\Gamma \vdash_{i, j} A$)

\begin{definition}[Forms of Judgment for Type and Term Equality]
    \begin{alignat*}{2}
        &\Gamma \vdash_{i, j} A = B &&\text{means ``$A$ and $B$ are the same type in stage $i$, universe level $j$''} \\
        &\Gamma \vdash_{i, j} t=u:A \hspace{2em} &&\text{means ``$t$ and $u$ are the same term of type $A$ in stage $i$, universe level $j$''}
    \end{alignat*}
\end{definition}

Judgmental equality also require specifying the stage and universe level. They also come with the presupposition that the two expressions in comparison are from the same stage and universe.

\begin{example}[Ill-formed Judgment]
    $\Gamma \vdash_{0, 0} \Nat_0 = \Nat_1$ is not well formed. Although $\Gamma \vdash_{0, 0} \Nat_0$ is believable, $\Gamma \vdash_{0, 0} \Nat_1$ is not believable. So we cannot judge the equality of the two types. 
\end{example}




\subsection{Familiar Inference Rules in the Context of 2LTT}
Here we look at some inference rules that are common in many type theories, but now they involve stage and universe level. 

\subsubsection{Context}
    We always have access to the empty context, as represented by the axiom:
    
    \begin{mathparpagebreakable}
        \inferrule*[lab=empty-cxt]
             {\\}
             {\emptycon \vdash}
    \end{mathparpagebreakable}

    If a type $A$ can be derived from a context $\Gamma$, we can extend $\Gamma$ with the type $A$

    \begin{mathparpagebreakable}
        \inferrule*[lab=cxt-extension]
             {\Gamma \vdash \\ \Gamma \vdash_{i,j} A}
             {\Gamma \ext A \vdash}
    \end{mathparpagebreakable}

    An interesting observation is that types from all stages and universe levels are treated equally with extending context. A type in stage 0 extends the context in the same way as a type in stage 1.

\subsubsection{Substitutions}
As mentioned in in the previous section on judgments, we will only employ the weakening substitution for De Bruijn indices. Nonetheless, it is important to mention that substitutions have an identity before covering weakening substitutions:

\begin{mathparpagebreakable}
    \inferrule*[lab=identity-sub]
        {\Gamma \vdash}
        {\Gamma \vdash\id:\Gamma}

    \inferrule*[lab=type-id-sub]
        {\Gamma \vdash_{i,j} A}
        {\Gamma \vdash_{i,j} A[\id]=A}

    \inferrule*[lab=term-id-sub]
        {\Gamma \vdash_{i,j} t:A}
        {\Gamma \vdash_{i,j} t[\id]=t:A}
\end{mathparpagebreakable}

Since we only extend a type to the context through \textsc{cxt-extension} rather than a type-term pair, we need to refer to the terms in our context. To do so, we can derive the \emph{zero variable} $\q$ and the \emph{weakening substitution} $\p$ when extending a context with an arbitrary type:

\begin{mathparpagebreakable}
    \inferrule*[lab=sub-first-proj]
        {\Gamma \vdash_{i,j}A}
        {\Gamma\ext A\vdash\p:\Gamma}

    \inferrule*[lab=sub-second-proj]
        {\Gamma \vdash_{i,j} A}
        {\Gamma\ext A\vdash\q: A[\p]}
\end{mathparpagebreakable}

From the rule \textsc{sub-second-proj}, the term $\q$ refers to a term in the context $\Gamma\ext A$, in particular, with type $A$ that we have just extended. In general, the term $\q$ always points to the most recently extended type. Similarly, the weakening substitution $\p$ that is applied to the type $A[\p]$ under the context $\Gamma\ext A$ means that $A[\p]$ refers to the type inside the context (namely, the most recently extended one) rather than a newly formed type. Therefore, if we continue to extend more types to the context, we need to shift the De Bruijn index $\q$ with these substitution rules:

\begin{mathparpagebreakable}
    \inferrule*[lab=type-sub]
        {\Delta\vdash_{i,j}A\\\Gamma\vdash\sigma:\Delta}
        {\Gamma\vdash_{i,j}A[\sigma]}

    \inferrule*[lab=term-sub]
        {\Delta\vdash_{i,j}t:A\\\Gamma\vdash\sigma:\Delta}
        {\Gamma\vdash_{i,j}t[\sigma]:A[\sigma]}
\end{mathparpagebreakable}

For example, suppose we can form the type $A$ under context $\Gamma$ and the type $B$ under $\Gamma\ext A$, then we can derive the following De Bruijn indices, which we will use colors to match with their type in the context:

\begin{mathparpagebreakable}
\inferrule*[right=term-sub]{
\inferrule*[right=sub-second-proj]{
    \Gamma\vdash\textcolor{green}{A}
}{\Gamma\ext\textcolor{green}{A}\vdash\textcolor{green}{\q}:A[\p]}
\inferrule*[Right=sub-first-proj]{
    \Gamma\ext\textcolor{green}{A}\vdash B
}{\Gamma\ext\textcolor{green}{A}\ext B\vdash\p:\Gamma\ext\textcolor{green}{A}}
}{\Gamma\ext \textcolor{green}{A}\ext B\vdash \textcolor{green}{\q[\p]}:A[\p][\p]}

\inferrule*[right=sub-second-proj]{
\Gamma\ext A\vdash\textcolor{blue}{B}
}{\Gamma\ext A\ext\textcolor{blue}{B}\vdash\textcolor{blue}{\q}:B[\p]}
\end{mathparpagebreakable}

Next, as substitutions are mappings from one context to another, they can be composed. In addition, applying the composition of two substitutions is judgmentally equal to applying the first, then applying the second. Despite the fact that there are other properties on substitution composition such as associativity and composing substitutions on terms, they won't be used for the purpose of this paper. Nonetheless, we have these inference rules:

\begin{mathparpagebreakable}
    \inferrule*[lab=sub-composition]
        {\Delta\vdash\sigma:\Theta\\\Gamma\vdash\delta:\Delta}
        {\Gamma\vdash\sigma\circ\delta:\Theta}

    \inferrule*[lab=type-comp-sub]
        {\Theta\vdash_{i,j}A\\\Delta\vdash\delta:\Theta\\\Gamma\vdash\sigma:\Delta}
        {\Gamma\vdash_{i,j}A[\sigma\circ\delta]=A[\sigma][\delta]}
\end{mathparpagebreakable}


Lastly, substitutions can be extended with a term, and applying an extended substitution to a term is analogous to function application. For instance, if a funtion $f$ is defined as $\lambda\,x.\mathsf{body}$ for some term $\mathsf{body}$ where $x$ might appear in, then the function application $f\,a$ results the same expression as the substitution $\mathsf{body}[\id,a]$ where $(\id,a)$ is the identity substitution extended with the term $a$. Extending a substitution is rarely used in this paper other than for the elimination of naturals, but for completeness, we have these inference rules:


\begin{mathparpagebreakable}
    \inferrule*[lab=sub-extension]
        {\Gamma\vdash\sigma:\Delta\\\Gamma\vdash t:A[\sigma]}
        {\Gamma\vdash(\sigma,t):\Delta\ext A}

    \inferrule*[lab=sub-extension-\circ]
        {\Delta\vdash\sigma:\Theta\\\Delta\vdash_{i,j}t:A[\sigma]\\\Gamma\vdash\delta:\Delta}
        {\Gamma\vdash(\sigma,t)\circ\delta=(\sigma\circ\delta,t[\delta]):\Theta\ext A}
\end{mathparpagebreakable}

As we revisit the $\mathsf{double}_0$ example for type-checking and staging, we will intensively apply these inference rules on substitution along with the three staging operations.

%==============================================================================


\subsubsection{Universes}
    Both stage 0 and stage 1 supports hierarchy of universes. 2LTT chooses the Russell-style universes as the framework for implementing hierarchy.
    %and Tarski-style universes are two methods of implementing hierarchy of universes. 
    % \begin{itemize}
    %     \item \emph{Russell-style Universes}: 
    %     \item \emph{Tarski-style Universes}: Tarski-style universes are supplied with the addition of two operators ``$\El, \Code$''. In Tarski-style universes, a term $t:\U$ can be transformed into a type with $\El\,t$, and a type $A$ can be transformed into a term of $\U$ with $\Code A : \U$.
    % \end{itemize} 

    
    \begin{definition}[Russell-style Universe]
        \emph{Russell-style universes} have types as terms of universes. For example, it has $\Gamma \vdash A : \U$ where $A$ is a type. % (The Little Typer uses this approach). 
    \end{definition}
    
    The rule \textsc{universe} describes the hierarchy:
    $$\inferrule*[lab=universe]
                 {\\}
                 {\Gamma \vdash_{i,j+1} \U_j}$$

    Notice the index shift by 1 between the type we are judging and the universe we are making judgment in. The rule states that regardless of the stage, a universe is a term of the next largest universe. This is a reflection of the Russell-style universes.

    % 2LTT also includes a way to transform a type from a lowerlevel universe into a term of the universe type at the next level of universe, or vise versa.  
    % 2LTT also includes some characteristics of Tarski-style universe, these are most likely a consequence of the mathematical foundation of 2LTT, which uses category theory, where $\El$ and $\Code$ are natural transformation. As the paper ``Two-Level Type Theory and Application'' stated, these also serves to model Russell-style universes \cite{2ltt}.
    % \begin{mathparpagebreakable}
    %     \inferrule*[lab=universe-decoding]
    %          {\Gamma \vdash_{i,j+1} t : \U_j}
    %          {\Gamma \vdash_{i,j} \El\,t}
    %              
    %      \inferrule*[lab=universe-encoding]
    %          {\Gamma \vdash_{i,j} A}
    %          {\Gamma \vdash_{i,j+1} \Code\,A : \U_j}
    % \end{mathparpagebreakable}

%==============================================================================

% \subsubsection{$\Pi$-Types}
%     Formation of $\Pi$-type should feel familiar
%     \begin{mathparpagebreakable}
%         \inferrule*[lab=$\Pi$-formation]
%              {\Gamma \vdash_{i,j} A \\ \Gamma \ext A \vdash_{i,j} B}
%              {\Gamma \vdash_{i,j} \Pi\,A\,B}
%     \end{mathparpagebreakable}

%     The rule \textsc{$\Pi$-formation} states: Assume we have context $\Gamma$, type $A$ is derivable from $\Gamma$, and extending $\Gamma$ with $A$ would furthermore allow us to derive type $B$, then we can derive $\Pi\,A\,B$ from $\Gamma$.

%     \begin{example}[Inferring $\Pi\,\Nat\,\Nat$ from the empty context] We demonstrate how to derive the type $(\Pi\,\Nat\,\Nat)$ (in stage 0 universe level 0) from the empty context $\emptycon$:
%         \begin{mathparpagebreakable}
%             \inferrule*[right=$\Pi$-formation]
%             {
%                 \inferrule*[left=Nat-formation]
%                 { }
%                 {\emptycon \vdash_{0,0} \Nat} \\ 
%                 \inferrule*[right=Nat-formation]
%                 {
%                     \inferrule*[right=cxt-extension]
%                     {   
%                     \inferrule*[right=nat-formation]
%                         { }
%                         {\emptycon \vdash_{0, 0} \Nat}
%                     }
%                     {\emptycon \ext \Nat \vdash}
%                 }
%                 {\emptycon \ext \Nat \vdash_{0,0} \Nat}
%             }
%             {\emptycon \vdash_{0,0} \Pi\,\Nat\,\Nat}
%         \end{mathparpagebreakable}
%     \end{example}

    
%     This might come as a surprise, but the core 2LTT language syntax does not include the ``$\lambda$'' symbol. This means the notation in our examples are syntactic sugar for the core 2LTT language. 2LTT instead have a constructor $\lam$ that construct a term of $\Pi$-type:
%     \begin{mathparpagebreakable}
%         \inferrule*[lab=lam]
%              {\Gamma \ext A \vdash_{i,j} t : B}
%              {\Gamma \vdash_{i,j} \lam\,t : \Pi\,A\,B}
%     \end{mathparpagebreakable}

%     One should imagine the term $\lam\, t$ as an abstraction of $t$, where some subterm of $t$ of type $A$ are abstracted away to an argument. 

%     $\Pi$ still represents dependent type. However, because the conventional $\beta$-reduction and substitution is replaced by context substitution, it takes a different method to understand. Unfortunately we are not able to provide a more intuitive understanding of such procedure, it is an area that we would like to work on in the future.




%==============================================================================


\subsubsection{Natural Numbers ($\Nat$)}
Within the context of staged compilation 2LTT, we assume all universes across both stages have access to $\Nat$. $\Nat$ is implemented through Peano Arithmetic, meaning we have term $\zero$ and the term former $\suc$ at our disposal.

\begin{mathparpagebreakable}
    \inferrule*[lab=nat-formation]
         {\\}
         {\Gamma \vdash_{i,j} \Nat}

    \inferrule*[lab=zero]
         {\\}
         {\Gamma \vdash_{i,j} \zero : \Nat}

    \inferrule*[lab=suc]
         {\Gamma \vdash_{i,j} t : \Nat}
         {\Gamma \vdash_{i,j} \suc\,t : \Nat}
\end{mathparpagebreakable}

We emphasis that although $\Nat$ is a type in any stage and universe, two $\Nat$s are not the same when they are from different stage or universe by our judgments. Type and term equality require the presupposition of the two types coming from the same stage and universe level. We use the notation $\Nat_0$ and $\Nat_1$ to extinguish $\Nat$ from different stages.

We also have eliminator for $\Nat$ in the form of $\msf{NatElim}$. This eliminator also comes with $\beta$ reductions as inference rules so we can apply substitutions on a given target.

$$
\inferrule*[lab=nat-elim]{
    \Gamma \ext \Nat \vdash_{i,k} P \\\\
    \Gamma \vdash_{i,k} z : P[\id,\,\zero] \\\\
    \Gamma \ext \Nat \ext P \vdash_{i,k} s : P[ \p \circ \p,\,\suc\,( \q [ \p ])] \\\\
    \Gamma \vdash_{i,j} t : \Nat}
 {\Gamma \vdash_{i,k} \msf{NatElim}\,P\,z\,s\,t : P[\id,\,t]}
$$

\begin{mathparpagebreakable}
    \inferrule*[lab=zero-$\beta$]
         {\\}
         {\Gamma \vdash_{i,j} \msf{NatElim}\,P\,z\,s\,\zero = z : P[\id,\,\zero]}

    \inferrule*[lab=suc-$\beta$]
         {\\}
         {\Gamma \vdash_{i,j} \msf{NatElim}\,P\,z\,s\,(\suc\,t) = s[\id,\,t,\,\msf{NatElim}\,P\,z\,s\,t]
           : P[\id,\,\suc\,t]}
\end{mathparpagebreakable}

While the premises of \textsc{nat-elim} involve some menacing substitutions on the type $P$, the substitutions on the type $\Nat$ and its term formers are rather simple. Again, we describe them as inference rules:

\begin{mathparpagebreakable}
  \inferrule*[lab=nat-sub]
             {\Gamma \vdash \sigma : \Delta}
             {\Gamma \vdash_{i,j} \Nat[\sigma] = \Nat}
  
  \inferrule*[lab=zero-sub]
             {\Gamma \vdash \sigma : \Delta}
             {\Gamma \vdash_{i,j} \zero[\sigma] = \zero : \Nat}

  \inferrule*[lab=suc-sub]
             {\Gamma \vdash \sigma : \Delta \\ \Delta \vdash_{i,j} t : \Nat}
             {\Gamma \vdash_{i,j} (\suc\,t)[\sigma] = \suc\,(t[\sigma]) : \Nat}
\end{mathparpagebreakable}



As described in the section on De Bruijn indices, substitutions can be thought as function applications. Therefore, by applying the notation on dependently typed functions, we can the eliminator of $\mathsf{Nat}$ as follows:

$$
\begin{aligned}
    \mathsf{NatElim}&:(P:\mathsf{Nat}\to\mathsf{U}_{i,j})\\
    &\to P\,\mathsf{zero}\\
    &\to ((n:\mathsf{Nat})\to P\,n\to P\,(\mathsf{suc}\,n))\\
    &\to(t:\mathsf{Nat})\\
    &\to P\,t\\
\end{aligned}
$$

Therefore, $\mathsf{NatElim}$ takes a motive $P:\mathsf{Nat}\to\mathsf{U}_{i,j}$, a base $z:P\,\mathsf{zero}$, a step function $s:(n:\mathsf{Nat})\to P\,n\to P\,(\mathsf{suc}\,n)$, and a target $t:\mathsf{Nat}$ as arguments; then it eliminates the target to $\mathsf{Nat}\,P\,z\,s\,t:P\,t$. The rules of elimination is described as the $\beta$-reductions, which we can rephrase as:

$$
\begin{aligned}
    \textsf{zero-}\beta&:\mathsf{NatElim}\,P\,z\,s\,\mathsf{zero}=z\\
    \textsf{suc-}\beta&:\mathsf{NatElim}\,P\,z\,s\,(\mathsf{suc}\, n)=s\,n\,(\mathsf{NatElim}\,P\,z\,s\,n)\\
\end{aligned}
$$

\begin{example}[Addition and Multiplication]
    We can implement addition, $\mathsf{add}:\mathsf{Nat}\to\mathsf{Nat}$ between two naturals $a,b:\mathsf{Nat}$ by eliminating $a$ with $\mathsf{NatElim}$. More specifically, on each $\textsf{suc-}\beta$ rule, we wrap $b$ (i.e. the base) with $\mathsf{suc}$:

    $$
    \begin{aligned}
        \mathsf{add}&:\mathsf{Nat}\to\mathsf{Nat}\\
        \mathsf{add}&:\lambda\,a\,b.\,\mathsf{NatElim}\,(\lambda\,\_.\,\mathsf{Nat})\,b\,(\lambda\,\_\,n.\,\mathsf{suc}\,n)\,a\\
    \end{aligned}
    $$
    With $\mathsf{add}$ implemented using the $\mathsf{NatElim}$ from our 2LTT model, we can implement multiplication $\mathsf{mul}\,a\,b$ in a similar manner, by eliminating $a$ with $\mathsf{zero}$ as base case, and perform $\mathsf{add}\,b$ on each \textsc{suc-$\beta$} reduction. Therefore, we can rewrite the example from section 1.1 into well-typed metaprogram in coherenece with our inference rules on $\mathsf{Nat}$s and lifting:

    $$
    \begin{aligned}
        \mathsf{add}_0&:\mathsf{Nat}_0\to\mathsf{Nat}_0\to\mathsf{Nat}_0\\
        \mathsf{add}_0&:=\lambda\,a\,b.\,\mathsf{NatElim}_0\,(\lambda\,\_.\,\mathsf{Nat}_0)\,b\,(\lambda\,\_\,n.\,\mathsf{suc}_0\,n)\,a\\
        \\
        \mathsf{mul}_1&:\mathsf{Nat}_1\to{\Uparrow}\mathsf{Nat}_0\to{\Uparrow}\mathsf{Nat}_0\\
        \mathsf{mul}_1&:=\lambda\,a\,b.\,\mathsf{NatElim}_1\,(\lambda\,\_.\,{\Uparrow}\mathsf{Nat}_0)\,\langle \mathsf{zero}_0\rangle\,(\lambda\,\_\,n.\,\langle \mathsf{add}_0\,{\sim}b\,{\sim}n\rangle)\,a\\
        \mathsf{two}_1&:=\mathsf{suc}_1\,(\mathsf{suc}_1\,\mathsf{zero}_1)\\
        \\
        \mathsf{double}_0&:\mathsf{Nat}_0\to\mathsf{Nat}_0\\
        \mathsf{double}_0&:=\lambda\,x.\,{\sim}(\mathsf{mul}_1\,\mathsf{two}_1\,\langle x\rangle)\\
    \end{aligned}
    $$
    In a later section, we will stage this metaprogram rigorously through inference rules and get the same result from the introduction.
\end{example}



%==============================================================================






\subsection{Lifting, Quoting, and Splicing}
We now move on to the new addition 2LTT brings to the table, \emph{lifting}, \emph{quoting}, and \emph{splicing}. Of all the inference rules we introduced so far, the premise and conclusion does not modify the index $i$ that represents the stage. We are going to address that by introducing inference rules that allows that.

\subsubsection{Lifting}
We start with \emph{lifting}. Syntactically, we define the annotation ``$\Lift \blank$'' (``$\blank$'' is placeholder for specific type). Stage 0 types are the only valid arguments that can be used in place of ``$\blank$''. The behaviour of lifting is as follows:

\begin{mathparpagebreakable}
    \inferrule*[lab=lift]
             {\Gamma \vdash_{0,j} A}
             {\Gamma \vdash_{1,j} \Lift A}

  \inferrule*[lab=lift-sub]
             {\Gamma \vdash \sigma : \Delta \\ \Delta \vdash_{0,j} A}
             {\Gamma \vdash_{1,j} (\Lift A)[\sigma] = \Lift(A[\sigma])}             
\end{mathparpagebreakable}

We see that given a type $A$ in stage 0, universe level $j$, applying lift to $A$ creates a new type $\Lift A$ in stage 1 of the same universe level. 

\begin{example}[Examples of Lifted Types]
    $\Lift\Nat_0, \Lift(\Pi\,\Nat_0\,\Nat_0)$ are examples of types that are in stage 1.
\end{example}

\begin{remark}[Lift and Native types are not the same]
    Both stage 0 and 1 has the type $\Nat$, denoted $\Nat_0, \Nat_1$. And we have
    \begin{mathparpagebreakable}
        \inferrule*
             {\Gamma \vdash_{0,0} \Nat_0}
             {\Gamma \vdash_{1,0} \Lift \Nat_0}

        \inferrule*
            { }
            {\Gamma \vdash_{1, 0} \Nat_1}
    \end{mathparpagebreakable}
        However, $\Lift \Nat_0$ is \emph{not} the same type as $\Nat_1$
\end{remark}

\begin{example}[Not Lifted Type]
    The type $(\Pi\,\Lift \Nat_0\,\Lift \Nat_0)$ is not a lifted type. It is a $\Pi$-type with domain and co-domain being lifted types. We demonstrate this with the inference tree below:
    \begin{mathparpagebreakable}
        \inferrule*[right=$\Pi$]
        {   
            \inferrule*[left=Lift]
            {
                \inferrule*[left=Nat]
                { }
                {\emptycon \vdash_{0, 0} \Nat_0}
            }
            {\emptycon \vdash_{1, 0} \Lift \Nat_0} \\ 
            \inferrule*[right=lift]
            {
                \inferrule*[right=nat]
                {
                    \inferrule*[right=cxt-extension]
                    {
                        \inferrule*[right=Lift]
                        {
                            \inferrule*[right=nat]
                            { }
                            {\emptycon \vdash_{0, 0} \Nat_0}
                        }
                        {\emptycon \vdash_{1, 0} \Lift \Nat_0}
                    }
                    {\emptycon \ext \Lift \Nat_0 \vdash}
                }
                {\emptycon \ext \Lift \Nat_0 \vdash_{0, 0} \Nat_0}
            }
            {\emptycon \ext \Lift \Nat_0 \vdash_{1, 0} \Lift \Nat_0}
        }
        {\emptycon \vdash_{1, 0} \Pi\,\Lift \Nat_0\,\Lift \Nat_0}
    \end{mathparpagebreakable}
    As we see at the bottom of the inference tree, the final step of inference is not \textsc{Lift}, so the type $\Pi, \Lift \Nat_0 \, \Lift \Nat_0$ is not a lifted type, but a stage 1 $\Pi$ type. 
\end{example}

\begin{remark}\label{lift-not-equal-native}
    We \emph{do not} believe the following judgment is true:
    $$\Gamma \vdash_{1, j} \Lift(\Pi\,A\,B) = \Pi\,\Lift A\,\Lift B$$
    However, as we will show in later section, there is a way to transform a term of type $\Pi\, \Lift A\,\Lift B$ into a lifted type. 
\end{remark}
\smallskip
\begin{remark}[Lift v.s. Stage Index]
   Let $A$ be a type that exists in both stages, denoted $A_0$ and $A_1$. We emphasis $\Lift A_0 \neq A_1$. 
\end{remark}
\smallskip

In another word, a lifted type $A$ from stage 0 is not equal to the equivalent to the same type $A$ native to stage 1. This is a small but important detail. 

\smallskip

\begin{remark}[Lifted Types Do Not Have Eliminator]
    There is \emph{no} general elimination rule for $\Lift A$.
\end{remark}

\begin{example}
    Take $\Bool_1$ and $\Lift \Bool_0$. Where $\Bool_{1/0}$ are the respective boolean type of stage 1 and 0 
    
    Unlike $\Bool_1$, $\Lift \Bool_0$ does not have eliminator. As a result, a function of type $\Pi\, \Lift \Bool_0\, \Bool_1$ would have to be a constant function, because there are no stage 1 eliminator that can inspect the internal of a term of type $\Lift \Bool_0$, and all eliminator eliminates a term to another term in the same stage.

    One might attempt to counter with the following function:

    $$f := \lambda b.\,\, \msf{if}\,\, \spl b\,\, \msf{then}\,\, \qut{\msf{True}_0}\,\, \msf{else}\,\, \qut{\msf{False}_0}$$
    
    However, this function outputs terms of type $\Lift \Bool_0$, which as we mentioned in $remark$ \ref{lift-not-equal-native}, is not the same type as $\Bool_1$.
\end{example}

No eliminator for $\Lift A$ will restrict terms we can produce, and a trade-off for the program to behave in a predictable manner. 





\subsubsection{Quoting}
We now introduce \emph{quoting}, used to create a term of lifted type. Quoting is assigned the annotation $\qut{\blank}$, (``$\blank$'' is placeholder for specific term). Its inference rule is as followed

\begin{mathparpagebreakable}
    \inferrule*[lab=quote]
             {\Gamma \vdash_{0,j} t : A}
             {\Gamma \vdash_{1,j} \qut{t} : \Lift A} 


  \inferrule*[lab=quote-sub]
             {\Delta \vdash_{0,j} t : A \\ \Gamma \vdash \sigma : \Delta}
             {\Gamma \vdash_{1,j} \qut{t}[\sigma] = \qut{t[\sigma]} : \Lift (A[\sigma])}

\end{mathparpagebreakable}

Given a term $t$ of type $A$ in stage 0, quoting $t$ to $\qut{t}$ gives us a term of type $\Lift A$ in stage 1.

\begin{example}[Examples of Quoted Terms]
    Some examples of quoted terms are
    \begin{itemize}
        \item $\qut{\zero_0}$ has type $\Lift \Nat_0$
        \item $\qut{\lambda x . \suc\,x }$ has type $\Lift (\Nat_0 \to \Nat_0)$
    \end{itemize}
\end{example}

\subsubsection{Splicing}
Lastly we have \emph{splicing}. Splicing has the annotation $\spl \blank$ (``$\blank$'' is placeholder for specific term). Splice also has the highest precedence, even higher than function application. So $\spl\, f \,x$ parses to $(\spl\,f)\,x$. Splicing comes with the inference rule:

\begin{mathparpagebreakable}
\inferrule*[lab=splice]
             {\Gamma \vdash_{1,j} t : \Lift A}
             {\Gamma \vdash_{0,j} \spl{t} : A}
\end{mathparpagebreakable}

Splice is used to bring a stage 0 term that has been lifted to stage 1 back down to 0.

\begin{remark}
    We can quote any term $t$ of type $A$ if $A$ is a type in stage 0.
    However, not every term $s$ of type $B$ in stage 1 can be spliced.
    For example, $\zero_1$ cannot be spliced because $\zero_1$ is of type $\Nat_1$, and $\Nat_1$ is not a lifted type. 
\end{remark}

\subsubsection{Interaction Between Quoting and Splicing}
There are two more inference rules that tells us $\qut{\blank}$ and $\spl \blank$ are inverses of each other:
\begin{mathparpagebreakable}
    \inferrule*[lab=quote-splice]
             {\Gamma \vdash_{1,j} t : \Lift A}
             {\Gamma \vdash_{1,j} \qut{\spl{t}} = t : \Lift A}

    \inferrule*[lab=splice-quote]
             {\Gamma \vdash_{0,j} t : A}
             {\Gamma \vdash_{0,j} \spl{\qut{t}} = t : A}
\end{mathparpagebreakable}

\begin{example}[Elimination]
    The term $\suc\,(\suc\,\zero_0)$ is judgmentally the same $\Nat_0$ as $\spl \qut{\suc\,(\suc\,\zero_0)}$
\end{example}

This makes intuitive sense. If no computation is done in between lifting and splicing, we should expect to get back the same term. 

% An interesting observation is the similarity between these two inference rules and $\eta$-rules for other types:

% \begin{mathparpagebreakable}
%     \inferrule*[lab=$\Sigma$-$\eta$ (2LTT)]
%              {\Gamma \vdash_{i,j} t : \Sigma\,A\,B}
%              {\Gamma \vdash_{i,j} (\fst\,t,\,\snd\,t) = t : \Sigma\,A\,B}
             
%     \inferrule*[lab=$\Sigma$-$\eta$ (The Little Typer)]
%             {\Gamma \vdash p_1 \equiv p_2 : (\Sigma\,([x \,A])\,D)}
%             {\Gamma \vdash p_1 \equiv (\mathsf{cons}\,(\mathsf{car}\,p_2)\,(\mathsf{cdr}\,p_2)) : (\Sigma\,([x\, A])\, D)}

%     \inferrule*[lab=FunSame-$\eta$ (The Little Typer)]
%             {x \notin dom(\Gamma) \\ \Gamma \vdash f_1 \equiv f_2 : (\Pi\,([x \,Arg])\, R)}
%             {\Gamma \vdash f_1 \equiv (\lambda\,(x)\,(f_2 \,x)) : (\Pi\,([x \,Arg])\, R)}
% \end{mathparpagebreakable}

% We thus suggests that splicing then quoting ($\qut{\spl \blank}$), and quoting then splicing ($\spl \qut{\blank}$) are the $\eta$-rules for staging annotation.

One may ask why we have two annotations $\qut{\blank}$ and $\spl$ for terms, which are inverses of each other, but only one annotation, the $\Lift$ for types? To answer this, we consider the origin of 2LTT where it was intended to make meta-theoretical results of homotopy type theory internal \cite{2ltt}. We now understand stage 0 being the object-level and stage 1 being the meta-level. There is usually a way of representing object or concepts of the object-level in meta-level. However, with object-level usually being simpler, there are concepts in the meta-level which cannot be represented in the object-level. To make an analogy, $\mathbb N$ in meta-level might be defined as a subset of $\mathbb R$, while $\mathbb N$ in object-level is defined using Peano Arithmetic. In such situation, we cannot guarantee the correctness of bringing $\mathbb N$ from meta-level down to object level. 

In fact, $\qut{\spl \blank}$ is not the identity function on all terms $t$ in stage 1, but only terms like $\qut{s}$ that were lifted from stage 0. The inventors of 2LTT phrased it as such:
\begin{quote}
    One intuition for the two levels is as follows: from a type in HoTT [Homotopy Type Theory, the inner type theory], we can extract a statement that can be phrased in the meta-theory. From a meta-theoretical statement \emph{about} HoTT, it is not always possible to construct a type. Thus, we can convert inner types into outer one, but not always vice versa \cite{2ltt}.
\end{quote}



\section{Applying Judgments and Inference Rules}
In this section, we provide examples that employ the inference rules as presented in the previous section. First, we apply the inference rules to derive the type of $\mathsf{mul}_1$ under the assumption that $\mathsf{add}_0$ has type $\mathsf{Nat}_0\to\mathsf{Nat}_0\to\mathsf{Nat}_0$. The type of $\mathsf{add}_0$ is derived in the same manner since $\mathsf{mul}_1$ and $\mathsf{add}_0$ are both defined with $\mathsf{NatElim}$. Second, we use the inference rules for lifting to prove the isomorphism between lifted function types and function types where the domain and codomain are lifted. This isomorphism property is useful as it can optimise the implementation of the 2LTT model.


\subsection{Type Derivation for $\mathsf{mul}_1$}

Back to the definition of $\mathsf{mul}_1$:

$$
\mathsf{mul}_1:=\lambda\,a\,b.\,\mathsf{NatElim}_1\,(\lambda\,\_.\,{\Uparrow}\mathsf{Nat}_0)\,\langle \mathsf{zero}_0\rangle\,(\lambda\,\_\,n.\,\langle \mathsf{add}_0\,{\sim}b\,{\sim}n\rangle)\,a\\
$$

The ultimate goal for this section is to derive the judgment that $\mathsf{mul}_1$ has the type $\mathsf{Nat}_1\to{\Uparrow}\mathsf{Nat}_0\to{\Uparrow}\mathsf{Nat}_0$ from the empty context. Since the definition for $\mathsf{mul}_1$ uses the function $\mathsf{add}_0$, we will assume the type of $\mathsf{add}_0$. In other words, let's add the following inference rule.

$$
\inferrule*[lab=add]
{\Gamma \vdash_{00} a:\mathsf{Nat}_0\\\Gamma \vdash_{00} b:\mathsf{Nat}_0}
{\Gamma \vdash_{00} \mathsf{add}_0\,a\,b:\mathsf{Nat}_0}
$$

Now, we have enough inference rules to derive the type of $\mathsf{mul}_1$ from the empty context. To begin, we will derive the arguments for the function $\mathsf{mul}_1$ through context extensions.

$$
\inferrule*[Right=nat-sub]{
    \inferrule*[Right=type-comp-sub]{
        \inferrule*[Right=term-sub]{
            \inferrule*[right=sub-second-proj]{
                \inferrule*[Right=nat-formation]{ }{\emptycon\vdash_{10}\textcolor{green}{\Nat_1}}
            }{\emptycon\ext\textcolor{green}{\Nat_1}\vdash_{10}\textcolor{green}{\q}:\Nat_1[\p]}
            \inferrule*[Right=sub-first-proj]{
                \inferrule*[Right=lift]{
                    \inferrule*[Right=nat-formation]{
                        \inferrule*[Right=cxt-extension]{
                            \inferrule*[Right=nat-formation]{ }{\emptycon\vdash_{10}\textcolor{green}{\Nat_1}}
                        }{\emptycon\ext\textcolor{green}{\Nat_1}\vdash}
                    }{\emptycon\ext\textcolor{green}{\Nat_1}\vdash_{00}\Nat_0}
                }{\emptycon\ext\textcolor{green}{\Nat_1}\vdash_{10}\Lift\Nat_0}
            }{\emptycon\ext\textcolor{green}{\Nat_1}\ext\Lift\Nat_0\vdash\textcolor{green}{\p}:\emptycon\ext\textcolor{green}{\Nat_1}}
        }{\emptycon\ext\textcolor{green}{\Nat_1}\ext\Lift\Nat_0\vdash_{10}\textcolor{green}{\q[\p]}:\Nat_1[\p][\p]}
    }{\emptycon\ext\textcolor{green}{\Nat_1}\ext\Lift\Nat_0\vdash_{10}\textcolor{green}{\q[\p]}:\Nat_1[\p\circ\p] }
}{\emptycon\ext\textcolor{green}{\Nat_1}\ext{\Lift\Nat_0}\vdash_{10}\textcolor{green}{\q[\p]}:\Nat_1 }
$$

$$
\inferrule*[Right=sub-second-proj]{
    \inferrule*[Right=lift]{
        \inferrule*[Right=nat-formation]{
            \inferrule*[Right=cxt-extension]{
                \inferrule*[Right=nat-formation]{ }{\emptycon\vdash_{10}{\Nat_1}}
            }{\emptycon\ext{\Nat_1}\vdash}
        }{\emptycon\ext{\Nat_1}\vdash_{00}\Nat_0}
    }{\emptycon\ext{\Nat_1}\vdash_{10}\textcolor{blue}{\Lift\Nat_0}}
}{\emptycon\ext{\Nat_1}\ext\textcolor{blue}{\Lift\Nat_0}\vdash\textcolor{blue}{\q}:\Lift\Nat_0}
$$

From the two derivations above, we obtain the terms $\textcolor{green}{\q[\p]:\Nat_1}$ and the term $\textcolor{blue}{\q:\Lift\Nat_0}$ under the context $\emptycon\ext\textcolor{green}{\Nat_1}\ext\textcolor{blue}{\Lift\Nat_0}$. These will become the arguments to our function $\mathsf{mul}_1$ in the form of De Bruijn indices, with first argument, $\textcolor{green}{\q[\p]:\Nat_1}$, being the target of our natural elimination. Notice our derivation also implies that $\emptycon\ext\Nat_1\ext\Lift\Nat_0$ is a context, thus we will denote this context as $\Gamma$ for the rest of the section for brevity.

As per the definition of $\mathsf{mul}_1$, it consists of terms to be passed to $\NatElim$ as arguments. We start with the derivation of $(\lambda\,\_.\Lift\Nat_0)$, which we will refer to as the motive.

$$
\inferrule*[Right=lift]{
    \inferrule*[Right=nat-formation]{
        \inferrule*[Right=cxt-extension]{
            \inferrule*[Right=nat-formation]{
                \Gamma\vdash
            }{\Gamma\vdash_{10}\Nat_1}
        }{\Gamma\ext\Nat_1\vdash}
    }{\Gamma\ext\Nat_1\vdash_{00}\Nat_0}
}{\Gamma\ext\Nat_1\vdash_{10}\Lift\Nat_0}
$$

Since we have derived $\Lift\Nat_0$ as our motive under the context $\Gamma\ext\Nat_1$, both the base and step should produce a $\Lift\Nat_0$ with some substitutions according to the inference rule \textsc{nat-elim}. Thus, we continue our derivation with the base, $\langle \mathsf{zero}_0\rangle$. We will reuse previously derived judgments as assumptions and apply equality judgments directly to our expressions.

$$
\inferrule*[Right=zero-sub]{
    \inferrule*[Right=quote-sub]{
        \inferrule*[Right=term-sub]{
            \inferrule*[right=sub-extension]{
                \inferrule*[right=identity-sub]{
                    \Gamma\vdash
                }{\Gamma\vdash\id:\Gamma}\\
                \inferrule*[Right=type-id-sub]{
                    \inferrule*[Right=zero]{
                        \Gamma\vdash
                    }{\Gamma\vdash_{10}:\mathsf{zero}_1:\Nat_1}
                }{\Gamma\vdash_{10}\mathsf{zero}_1:\Nat_1[\id]}
            }{\Gamma\vdash(\id,\mathsf{zero}_1):\Gamma\ext\Nat_1}
            \,
            \inferrule*[Right=lift]{
                \inferrule*[Right=zero]{
                    \Gamma\ext\Nat_1\vdash
                }{\Gamma\ext\Nat_1\vdash_{00}\mathsf{zero}_0:\Nat_0}
            }{\Gamma\ext\Nat_1\vdash_{10}\langle \mathsf{zero}_0\rangle:\Lift\Nat_0}
        }{\Gamma\vdash_{10}\langle\mathsf{zero}_0\rangle[\id,\mathsf{zero}_1]:(\Lift\Nat_0)[\id,\mathsf{zero}_1]}
    }{\Gamma\vdash_{10}\langle \mathsf{zero}_0[\id,\mathsf{zero}_1]\rangle:(\Lift\Nat_0)[\id,\mathsf{zero}_1]}
}{\Gamma\vdash_{10}:\langle \mathsf{zero}_0\rangle:(\Lift\Nat_0)[\id,\mathsf{zero}_1]}
$$

When our target of elimination is $\mathsf{zero}_1:\Nat_1$, the multiplication function should return $\langle \mathsf{zero}_0\rangle$. Otherwise, the target is be nested with $\mathsf{suc}_1$. Thus, to perform multiplication with our argument $\textcolor{blue}{\q:\Lift\Nat_0}$, we add $\textcolor{blue}{\q}$ on each step. 

Since the output of the step function is a $(\Lift\Nat_0)[\p\circ\p,\suc(\q[\p])]$ as per the inference rule \textsc{nat-elim}, we first simplify this type by deriving $\Nat_0[\p\circ\p,\suc(\q[\p])]=\Nat_0$. Next, as the step function uses $\mathsf{add}_0$ according to the definition of $\mathsf{mul}_1$, we derive the first argument of $\mathsf{add}_0$ as $\textcolor{blue}{\q:\Lift\Nat_0}$ in our first derivation. However, we need to shift its De Bruijn index as the context is extended for the step function. Lastly, we apply the inference rule \textsc{add} as described in the beginning of the section.


$$
% \inferrule*[Right=nat-sub]{
%     \inferrule*[Right=suc-sub]{
%         \inferrule*[Right=sub-extension-\circ]{
%             \inferrule*[Right=sub-composition]{
%                 \inferrule*[right=sub-extension]{
%                     \inferrule*[right=nat-sub]{
%                         \inferrule*[Right=suc]{
%                             \inferrule*[Right=nat-sub]{
%                                 \inferrule*[Right=sub-second-proj]{
%                                     \Gamma\vdash_{10}\Nat_1
%                                 }{\Gamma\ext\Nat_1\vdash_{10}\q:\Nat[\p]}
%                             }{\Gamma\ext\Nat_1\vdash_{10}\q:\Nat_1}
%                         }{\Gamma\ext\Nat_1\vdash_{10}\suc_1\,\q:\Nat_1}
%                     }{\Gamma\ext\Nat_1\vdash_{10}\suc_1\,\q:\Nat_1[\p]}\,
%                     \inferrule*[Right=sub-first-proj]{
%                         \Gamma\vdash_{10}\Nat_1
%                     }{\Gamma\ext\Nat_1\vdash\p:\Gamma}
%                 }{\Gamma\ext\Nat_1\vdash(\p,\suc\,\q):\Gamma\ext\Nat_1}\,
%                 \inferrule*[Right=sub-first-proj]{
%                     \Gamma\ext\Nat_1\vdash_{10}\Lift\Nat_0
%                 }{\Gamma\ext\Nat_1\ext\Lift\Nat_0\vdash\p:\Gamma\ext\Nat}
%             }{\Gamma\ext\Nat_1\ext\Lift\Nat_0\vdash(\p,\suc\,\q)\circ\p:\Gamma\ext\Nat_1}
%         }{\Gamma\ext\Nat_1\ext\Lift\Nat_0\vdash(\p\circ\p,(\suc\,\q)[\p]):\Gamma\ext\Nat_1}
%     }{\Gamma\ext\Nat_1\ext\Lift\Nat_0\vdash(\p\circ\p,\suc(\q[\p])):\Gamma\vdash\Nat_1}
% }{\Gamma\ext\Nat_1\ext\Lift\Nat_0\vdash_{00}\Nat_0[\p\circ\p,\suc(\q[\p])]=\Nat_0\quad (1)}
$$

$$
\inferrule*[right=splice]{
    \inferrule*[Right=nat-sub]{
        \inferrule*[Right=lift-sub]{
            \inferrule*[Right=term-sub]{
                {\Gamma\vdash_{10}\textcolor{blue}{\q}:\Lift\Nat_0}
                \inferrule*[Right=sub-composition]{
                    \inferrule*[right=sub-first-proj]{
                        \Gamma\ext\Nat_1\vdash_{10}\Lift\Nat_0
                    }{\Gamma\ext\Nat_1\ext\Lift\Nat_0\vdash\p:\Gamma\ext\Nat_1}
                    \inferrule*[Right=sub-first-proj]{
                        \Gamma\vdash\Nat_1
                    }{\Gamma\ext\Nat_1\vdash\p:\Gamma}
                }{\Gamma\ext\Nat_1\ext\Lift\Nat_0\vdash\p\circ\p:\Gamma}
            }{\Gamma\ext\Nat_1\ext\Lift\Nat_0\vdash_{10}\textcolor{blue}{\q[\p\circ\p]}:(\Lift\Nat_0)[\p\circ\p]}
        }{\Gamma\ext\Nat_1\ext\Lift\Nat_0\vdash_{10}\textcolor{blue}{\q[\p\circ\p]}:\Lift(\Nat_0[\p\circ\p])}
    }{\Gamma\ext\Nat_1\ext\Lift\Nat_0\vdash_{10}\textcolor{blue}{\q[\p\circ\p]}:\Lift\Nat_0}
}{\Gamma\ext\Nat_1\ext\Lift\Nat_0\vdash_{00}\spl(\textcolor{blue}{\q[\p\circ\p]}):\Nat_0}
$$

$$
\inferrule*[Right=lift-sub]{
\inferrule*[Right=quote]{
\inferrule*[Right=(1)]{
\inferrule*[Right=add]{
{\Gamma\ext\Nat_1\ext\Lift\Nat_0\vdash_{00}\spl(\textcolor{blue}{\q[\p\circ\p]}):\Nat_0}
\inferrule*[Right=splice]{
\inferrule*[Right=nat-sub]{
\inferrule*[Right=lift-sub]{
\inferrule*[Right=sub-second-proj]{
\Gamma\ext\Nat_1\vdash_{10}\Lift\Nat_0
}{\Gamma\ext\Nat_1\ext\Lift\Nat_0\vdash_{10}\q:(\Lift\Nat_0)[\p]}
}{\Gamma\ext\Nat_1\ext\Lift\Nat_0\vdash_{10}\q:\Lift(\Nat_0[\p])}
}{\Gamma\ext\Nat_1\ext\Lift\Nat_0\vdash_{10}\q:\Lift\Nat_0}
}{\Gamma\ext\Nat_1\ext\Lift\Nat_0\vdash_{00}\spl\q:\Nat_0}
}{\Gamma\ext\Nat_1\ext\Lift\Nat_0\vdash_{00}\mathsf{add}_0\,\spl(\textcolor{blue}{\q[\p\circ\p]})\,\spl\q:\Nat_0}
}{\Gamma\ext\Nat_1\ext\Lift\Nat_0\vdash_{00}\mathsf{add}_0\,\spl(\textcolor{blue}{\q[\p\circ\p]})\,\spl\q:\Nat_0[\p\circ\p,\suc(\q[\p])]}
}{\Gamma\ext\Nat_1\ext\Lift\Nat_0\vdash_{10}\qut{\mathsf{add}_0\,\spl(\textcolor{blue}{\q[\p\circ\p]})\,\spl\q}:\Lift(\Nat_0[\p\circ\p,\suc(\q[\p])])}
}{\Gamma\ext\Nat_1\ext\Lift\Nat_0\vdash_{10}\qut{\mathsf{add}_0\,\spl(\textcolor{blue}{\q[\p\circ\p]})\,\spl\q}:(\Lift\Nat_0)[\p\circ\p,\suc(\q[\p])]}
$$

Now we have derived all the arguments for $\NatElim$, we follow the definition of $\mathsf{mul}_1$ and derive the following:

$$
\inferrule*[]{
\inferrule*[Right=nat-sub]{
\inferrule*[Right=lift-sub]{
\inferrule*[Right=nat-elim]{
\Gamma\ext\Nat_1\vdash_{10}\Lift\Nat_0\\\\
\Gamma\vdash_{10}:\langle \mathsf{zero}_0\rangle:(\Lift\Nat_0)[\id,\mathsf{zero}_1]\\\\
\Gamma\ext\Nat_1\ext\Lift\Nat_0\vdash_{10}\qut{\mathsf{add}_0\,\spl(\textcolor{blue}{\q[\p\circ\p]})\,\spl\q}:(\Lift\Nat_0)[\p\circ\p,\suc(\q[\p])]\\\\
\Gamma\vdash_{10}\textcolor{green}{\q[\p]}:\Nat_1
}{\Gamma\vdash_{10}\NatElim\,\Lift\Nat_0\,\langle \mathsf{zero}_0\rangle\,\qut{\mathsf{add}_0\,\spl(\textcolor{blue}{\q[\p\circ\p]})\,\spl\q}\,\textcolor{green}{\q[\p]}:(\Lift\Nat_0)[\id,\textcolor{green}{\q[\p]}]}
}{\Gamma\vdash_{10}\NatElim\,\Lift\Nat_0\,\langle \mathsf{zero}_0\rangle\,\qut{\mathsf{add}_0\,\spl(\textcolor{blue}{\q[\p\circ\p]})\,\spl\q}\,\textcolor{green}{\q[\p]}:\Lift(\Nat_0[\id,\textcolor{green}{\q[\p]}])}
}{\Gamma\vdash_{10}\NatElim\,\Lift\Nat_0\,\langle \mathsf{zero}_0\rangle\,\qut{\mathsf{add}_0\,\spl(\textcolor{blue}{\q[\p\circ\p]})\,\spl\q}\,\textcolor{green}{\q[\p]}:\Lift\Nat_0}
}{\emptycon\ext\textcolor{green}{\Nat_1}\ext\textcolor{blue}{\Lift\Nat_0}\vdash_{10}\NatElim\,\Lift\Nat_0\,\langle \mathsf{zero}_0\rangle\,\qut{\mathsf{add}_0\,\spl(\textcolor{blue}{\q[\p\circ\p]})\,\spl\q}\,\textcolor{green}{\q[\p]}:\Lift\Nat_0}
$$

In the last step of the derivation above, we apply the assumption that the context $\Gamma$ denotes $\emptycon\ext\textcolor{green}{\Nat_1}\ext\textcolor{blue}{\Lift\Nat_0}$ since the beginning of the section. The end result of our derivation shows that, for an $\textcolor{green}{a:\Nat_1}$ and $\textcolor{blue}{b:\Lift\Nat_0}$, the expression $\NatElim\,\Lift\Nat_0\,\qut{\mathsf{zero}_0}\,\qut{\mathsf{add}_0\,\spl\textcolor{blue}{b}\,\spl\q)}\,\textcolor{green}{a}$ has the type $\Lift\Nat_0$, where $\q$ is an argument in the step function. In other words, the step function is $\lambda\,\_.\,\lambda\,n.\qut{\mathsf{add}_0\,\spl\textcolor{blue}{b}\,\spl n}$. Ultimately, our derived expression matches the definition of $\mathsf{mul}_1$, thus $\mathsf{mul}_1$ has the type $\Nat_1\to\Lift\Nat_0\to\Lift\Nat_0$ as wanted. Therefore, we can now use the derivation to stage the metaprogram $\mathsf{double}_0:=\lambda\,x.\,\spl(\mathsf{mul}_1\,(\suc_1\,(\suc_1\,\mathsf{zero}_1))\,\qut x)$ through the equality judgments of $\NatElim$, which will result the same output as described in the introduction.



\subsection{Staging $\mathsf{double_0}$ Formally}

Bringing back the definition of $\mathsf{double}_0$:

$$
\begin{aligned}
    \mathsf{double}_0&:\mathsf{Nat}_0\to\mathsf{Nat}_0\\
    \mathsf{double}_0&:=\lambda\,x.\,{\sim}(\mathsf{mul}_1\,(\suc_1\,(\suc_1\,\mathsf{zero}_1))\,\langle x\rangle)\\
\end{aligned}
$$

Based on the definition above, we substitute $\textcolor{green}{\q[\p]}$ with the term $\suc_1\,(\suc_1\,\mathsf{zero}_1):\Nat_1$ and $\textcolor{blue}{\q[\p\circ\p]}$ with $\qut{x}:\Lift\Nat_0$. Thus, applying the inference rules, we stage the program as follows:

\begin{align*}
    &\lambda\,x.\,{\sim}(\mathsf{mul}_1\,(\suc_1\,(\suc_1\,\mathsf{zero}_1))\,\langle x\rangle)\\
    =\,&\lambda\,x.\,\spl(\NatElim\,\Lift\Nat_0\,\qut{\mathsf{zero}_0}\,\qut{\mathsf{add}_0\,\spl\qut x\,\spl\q}\,(\suc_1\,(\suc_1\,\mathsf{zero}_1)))\tag{definition of $\mathsf{mul}_1$}\\
    =\,&\lambda\,x.\,\spl(\NatElim\,\Lift\Nat_0\,\qut{\mathsf{zero}_0}\,\qut{\mathsf{add}_0\,x\,\spl\q}\,(\suc_1\,(\suc_1\,\mathsf{zero}_1)))\tag{\textsc{splice-quote}}\\
    =\,&\lambda\,x.\,\spl\qut{\mathsf{add}_0\,x\,\spl(\NatElim\,\Lift\Nat_0\,\qut{\mathsf{zero}_0}\,\qut{\mathsf{add}_0\,x\,\spl\q}\,(\suc_1\,\mathsf{zero}_1))}\tag{\textsc{suc-$\beta$}}\\
    =\,&\lambda\,x.\,\mathsf{add}_0\,x\,\spl(\NatElim\,\Lift\Nat_0\,\qut{\mathsf{zero}_0}\,\qut{\mathsf{add}_0\,x\,\spl\q}\,(\suc_1\,\mathsf{zero}_1))\tag{\textsc{splice-quote}}\\
    =\,&\lambda\,x.\,\mathsf{add}_0\,x\,\spl\qut{\mathsf{add}_0\,x\,\spl(\NatElim\,\Lift\Nat_0\,\qut{\mathsf{zero}_0}\,\qut{\mathsf{add}_0\,x\,\spl\q}\,\mathsf{zero}_1})\tag{\textsc{suc-$\beta$}}\\
    =\,&\lambda\,x.\,\mathsf{add}_0\,x\,(\mathsf{add}_0\,x\,\spl(\NatElim\,\Lift\Nat_0\,\qut{\mathsf{zero}_0}\,\qut{\mathsf{add}_0\,x\,\spl\q}\,\mathsf{zero}_1))\tag{\textsc{splice-quote}}\\
    =\,&\lambda\,x.\,\mathsf{add}_0\,x\,(\mathsf{add}_0\,x\,\spl\qut{\mathsf{zero}_0})\tag{\textsc{zero-$\beta$}}\\
    =\,&\lambda\,x.\,\mathsf{add}_0\,x\,(\mathsf{add}_0\,x\,\mathsf{zero}_0)\tag{\textsc{splice-quote}}\\
\end{align*}




\section{Discussion}
We have completed our main objective of applying and type-checking our metaprogram $\mathsf{double}_0$. Now we will discuss some properties of 2LTT that can serve as direction for further exploration.

\subsection{Isomorphism Property of Lifting and Quoting}
In mathematics, \emph{isomorphism} describes a bijective function that preserves ``properties''. The properties an isomorphism preserves can differ from context to context. 

% \begin{example}[Isomorphisms in Different Context]
%     In graph theory, an isomorphism between two finite graphs $G = (V, E), G' = (V', E')$ is a bijective function $f: V \to V'$ such that $\{v, u\} \in E \Leftrightarrow \{f(v), f(v)\} \in E'$. In this case, the property the isomorphism preserves is the "connectivity" of the graph.

%     In set theory, an isomorphism between two linearly ordered sets $(P, <), (Q, \prec)$ is a bijective function $f: P \to Q$ such that $a < b \Leftrightarrow f(a) \prec f(b)$. In this case, the property the isomorphism preserves is the "ordering" of the set.
% \end{example}


\begin{definition}[Isomorphic]
    If there is an isomorphism between two sets $A, B$, we call these two sets \emph{isomorphic}.
\end{definition}

We do not believe $\Lift (\Pi\,A\,B)$ is the same type as $\Pi\,(\Lift A)\,(\Lift B)$. However, it is possible to show that there two types are similar enough that we can consider them to be isomorphic. 

% \begin{notation}[Alternative Notation for Dependent Type]
%     We sometimes write $(x : A) \to B\,x$ as supposed to $\Pi\,A\,B$. $(x : A) \to B\,x$ means a function that has:
%     \begin{itemize}
%         \item Input of type $A$, with the input denoted $x$.
%         \item Output of type $Beeeee\,x$, where output type is depended on the input term value.
%     \end{itemize}
% \end{notation}

To set up the demonstration, we first define two functions.

\begin{definition}[Transformation between $\Lift((x : A) \to B\,x)$ and $(x : \Lift A) \to \Lift\,(B\,\spl x)$] We define two functions:
    \begin{align*}
        &pres_\to: \Lift((x : A) \to B\,x) \to ((x : \Lift A) \to \Lift\,(B\,\spl x)) \\
        &pres_\to := \lambda\,f.\,\,\lambda\,x.\,\,\qut{\spl f\,\spl x}\\
        &pres_\to^{-1}: ((x : \Lift A) \to \Lift\,(B\,\spl x)) \to \Lift((x : A) \to B\,x)\\
        &pres_\to^{-1} := \lambda\,f.\,\,\qut{\lambda\,x.\,\,\spl(f \qut{x})}
    \end{align*}
\end{definition}

\begin{remark}[$pres_\to$ and $pres_\to^{-1}$ don't modify the input function]
    We now demonstrate $pres_\to$ and $pres_\to^{-1}$ does not modify the input function's behaviour. We do so by showing it is possible to recover the original function through an inverse operation. \\
    Let $f: (x : \Lift A) \to \Lift\,(B\,\spl x)$. Consider the following reduction:\\
    \begin{align*}
      pres_\to (pres_\to^{-1} f)
      &=(\lambda\,f. \,\,\lambda\,x.\,\,\qut{\spl f\,\spl x})\,\, ((\lambda\,f.\,\,\qut{\lambda\,x.\,\,\spl(f\,\qut{x})})\,\, f) \\
      &= (\lambda\,f.\,\,\lambda\,x.\,\,\qut{\spl f\,\spl x})\,\, \qut{\lambda\,x.\,\,\spl(f\,\qut{x})} \tag{$\beta$-reduction}\\
      &= \lambda\,x.\,\,\qut{\spl \qut{\lambda\,x.\,\,\spl(f\,\qut{x})}\,\spl x} \tag{$\beta$-reduction}\\
      &= \lambda\,x.\,\,\qut{(\lambda\,x.\,\,\spl(f\,\qut{x}))\,\spl x} \tag{\textsc{splice-quote}}\\
      &= \lambda\,x.\,\,\qut{\spl(f\,\qut{\spl x})} \tag{$\beta$-reduction}\\
      &= \lambda\,x.\,\,\qut{\spl(f\, x)} \tag{\textsc{quote-splice}}\\
      &= \lambda\,x.\,\,(f\, x) \tag{\textsc{quote-splice}}\\
      &= f \tag{Function $\eta$-reduction}
    \end{align*}
\end{remark}


\smallskip
Now we are ready to demonstrate that the lifting process possesses isomorphic behaviour. For those more comfortable with mathematics, and isomorphism $\phi$ often satisfies the following structure: 
$$\phi(fa) = \phi(f)\phi(a)$$

Ideally, we would define for all $x$, $\phi(x) = \qut{x}$, and show $\qut{f\, a} = \qut{b} = \qut{f}\,\qut{a}$ to match the structure. However, because the expression  $(\qut{f}\,\qut{a})$ is undefined due to type mismatch, we achieve a similar structure by $pres_\to$ to $\qut{f}$ first, which doesn't change $f$'s behaviour as shown. With that, we will show the lifting process possesses properties of isomorphism.

\begin{example}[The effect of lifting]
    Let $f: (x: A) \to B\,x$, and $a: A, b: (B\,a)$ such that $f\,a = b$, then we have the following reduction:

    \begin{align*}
        (pres_\to \qut{f}) \qut{a} 
        &= ((\lambda\,f\,x.\,\,\qut{\spl f \,\, \spl x}) \,\, \qut{f}) \,\, \qut{a} \\
        &= (\lambda\,x.\,\,\qut{\spl \qut{f} \,\, \spl x}) \,\, \qut{a} \\
        &= \qut{\spl \qut{f} \,\, \spl \qut{a}} \\
        &= \qut{f \,\, a} \\
        &= \qut{b}
    \end{align*}
\end{example}






% \begin{remark}[$pres_\to$ and $pres_\to^{-1}$ Cancel out During Staging]
%     Let $f: (x : \Lift A) \to \Lift\,(B\,\spl x)$, the input type of $pres_\to^{-1}$, consider the following reduction:\\
%     \begin{align*}
%       pres_\to (pres_\to^{-1} f)
%       &=(\lambda\,f. \,\,\lambda\,x.\,\,\qut{\spl f\,\spl x})\,\, ((\lambda\,f.\,\,\qut{\lambda\,x.\,\,\spl(f\,\qut{x})})\,\, f) \\
%       &= (\lambda\,f.\,\,\lambda\,x.\,\,\qut{\spl f\,\spl x})\,\, \qut{\lambda\,x.\,\,\spl(f\,\qut{x})} \tag{$\beta$-reduction}\\
%       &= \lambda\,x.\,\,\qut{\spl \qut{\lambda\,x.\,\,\spl(f\,\qut{x})}\,\spl x} \tag{$\beta$-reduction}\\
%       &= \lambda\,x.\,\,\qut{(\lambda\,x.\,\,\spl(f\,\qut{x}))\,\spl x} \tag{\textsc{splice-quote}}\\
%       &= \lambda\,x.\,\,\qut{\spl(f\,\qut{\spl x})} \tag{$\beta$-reduction}\\
%       &= \lambda\,x.\,\,\qut{\spl(f\, x)} \tag{\textsc{quote-splice}}\\
%       &= \lambda\,x.\,\,(f\, x) \tag{\textsc{quote-splice}}\\
%       &= f \tag{Function $\eta$-reduction}
%     \end{align*}
    
%     This shows us 
% \end{remark}




\section{Conclusion}
We introduced staged compilation, more specifically, two-stage compilation as a useful way to write metaprograms that generate code with safety. With the special operations for moving between stages (namely lifting, quoting, and splicing) integrated to the typing rules, the 2LTT model provides not only guarantees the well-formedness of the generated output, but also support for dependent types on both stages. While this paper only covered the typing rules for type-checking a metaprogram, the staging algorithm (also known as the substitution calculus) can alse be formalised into typing rules through categorical logic (also known as abstract nonsense logic).

% Should attempt to explore the challenge of allowing type formers to form types using types from different stages. It is my suspicion that the challenge would be considering which universe would the produced type be in




\printbibliography

\end{document}

