We now move on to the new addition 2LTT brings to the table, \emph{lifting}, \emph{quoting}, and \emph{splicing}. Of all the inference rules we introduced so far, the premise and conclusion does not modify the index $i$ that represents the stage. We are going to address that by introducing inference rules that allows that.

\subsubsection{Lifting}
We start with \emph{lifting}. Syntactically, we define the annotation ``$\Lift \blank$'' (``$\blank$'' is placeholder for specific type). Stage 0 types are the only valid arguments that can be used in place of ``$\blank$''. The behaviour of lifting is as follows:

\begin{mathparpagebreakable}
    \inferrule*[lab=lift]
             {\Gamma \vdash_{0,j} A}
             {\Gamma \vdash_{1,j} \Lift A}

  \inferrule*[lab=lift-sub]
             {\Gamma \vdash \sigma : \Delta \\ \Delta \vdash_{0,j} A}
             {\Gamma \vdash_{1,j} (\Lift A)[\sigma] = \Lift(A[\sigma])}             
\end{mathparpagebreakable}

We see that given a type $A$ in stage 0, universe level $j$, applying lift to $A$ creates a new type $\Lift A$ in stage 1 of the same universe level. 

\begin{example}[Examples of Lifted Types]
    $\Lift\Nat_0, \Lift(\Pi\,\Nat_0\,\Nat_0)$ are examples of types that are in stage 1.
\end{example}

\begin{remark}[Lift and Native types are not the same]
    Both stage 0 and 1 has the type $\Nat$, denoted $\Nat_0, \Nat_1$. And we have
    \begin{mathparpagebreakable}
        \inferrule*
             {\Gamma \vdash_{0,0} \Nat_0}
             {\Gamma \vdash_{1,0} \Lift \Nat_0}

        \inferrule*
            { }
            {\Gamma \vdash_{1, 0} \Nat_1}
    \end{mathparpagebreakable}
        However, $\Lift \Nat_0$ is \emph{not} the same type as $\Nat_1$
\end{remark}

\begin{example}[Not Lifted Type]
    The type $(\Pi\,\Lift \Nat_0\,\Lift \Nat_0)$ is not a lifted type. It is a $\Pi$-type with domain and co-domain being lifted types. We demonstrate this with the inference tree below:
    \begin{mathparpagebreakable}
        \inferrule*[right=$\Pi$]
        {   
            \inferrule*[left=Lift]
            {
                \inferrule*[left=Nat]
                { }
                {\emptycon \vdash_{0, 0} \Nat_0}
            }
            {\emptycon \vdash_{1, 0} \Lift \Nat_0} \\ 
            \inferrule*[right=lift]
            {
                \inferrule*[right=nat]
                {
                    \inferrule*[right=cxt-extension]
                    {
                        \inferrule*[right=Lift]
                        {
                            \inferrule*[right=nat]
                            { }
                            {\emptycon \vdash_{0, 0} \Nat_0}
                        }
                        {\emptycon \vdash_{1, 0} \Lift \Nat_0}
                    }
                    {\emptycon \ext \Lift \Nat_0 \vdash}
                }
                {\emptycon \ext \Lift \Nat_0 \vdash_{0, 0} \Nat_0}
            }
            {\emptycon \ext \Lift \Nat_0 \vdash_{1, 0} \Lift \Nat_0}
        }
        {\emptycon \vdash_{1, 0} \Pi\,\Lift \Nat_0\,\Lift \Nat_0}
    \end{mathparpagebreakable}
    As we see at the bottom of the inference tree, the final step of inference is not \textsc{Lift}, so the type $\Pi, \Lift \Nat_0 \, \Lift \Nat_0$ is not a lifted type, but a stage 1 $\Pi$ type. 
\end{example}

\begin{remark}\label{lift-not-equal-native}
    We \emph{do not} believe the following judgment is true:
    $$\Gamma \vdash_{1, j} \Lift(\Pi\,A\,B) = \Pi\,\Lift A\,\Lift B$$
    However, as we will show in later section, there is a way to transform a term of type $\Pi\, \Lift A\,\Lift B$ into a lifted type. 
\end{remark}
\smallskip
\begin{remark}[Lift v.s. Stage Index]
   Let $A$ be a type that exists in both stages, denoted $A_0$ and $A_1$. We emphasis $\Lift A_0 \neq A_1$. 
\end{remark}
\smallskip

In another word, a lifted type $A$ from stage 0 is not equal to the equivalent to the same type $A$ native to stage 1. This is a small but important detail. 

\smallskip

\begin{remark}[Lifted Types Do Not Have Eliminator]
    There is \emph{no} general elimination rule for $\Lift A$.
\end{remark}

\begin{example}
    Take $\Bool_1$ and $\Lift \Bool_0$. Where $\Bool_{1/0}$ are the respective boolean type of stage 1 and 0 
    
    Unlike $\Bool_1$, $\Lift \Bool_0$ does not have eliminator. As a result, a function of type $\Pi\, \Lift \Bool_0\, \Bool_1$ would have to be a constant function, because there are no stage 1 eliminator that can inspect the internal of a term of type $\Lift \Bool_0$, and all eliminator eliminates a term to another term in the same stage.

    One might attempt to counter with the following function:

    $$f := \lambda b.\,\, \msf{if}\,\, \spl b\,\, \msf{then}\,\, \qut{\msf{True}_0}\,\, \msf{else}\,\, \qut{\msf{False}_0}$$
    
    However, this function outputs terms of type $\Lift \Bool_0$, which as we mentioned in $remark$ \ref{lift-not-equal-native}, is not the same type as $\Bool_1$.
\end{example}

No eliminator for $\Lift A$ will restrict terms we can produce, and a trade-off for the program to behave in a predictable manner. 





\subsubsection{Quoting}
We now introduce \emph{quoting}, used to create a term of lifted type. Quoting is assigned the annotation $\qut{\blank}$, (``$\blank$'' is placeholder for specific term). Its inference rule is as followed

\begin{mathparpagebreakable}
    \inferrule*[lab=quote]
             {\Gamma \vdash_{0,j} t : A}
             {\Gamma \vdash_{1,j} \qut{t} : \Lift A} 


  \inferrule*[lab=quote-sub]
             {\Delta \vdash_{0,j} t : A \\ \Gamma \vdash \sigma : \Delta}
             {\Gamma \vdash_{1,j} \qut{t}[\sigma] = \qut{t[\sigma]} : \Lift (A[\sigma])}

\end{mathparpagebreakable}

Given a term $t$ of type $A$ in stage 0, quoting $t$ to $\qut{t}$ gives us a term of type $\Lift A$ in stage 1.

\begin{example}[Examples of Quoted Terms]
    Some examples of quoted terms are
    \begin{itemize}
        \item $\qut{\zero_0}$ has type $\Lift \Nat_0$
        \item $\qut{\lambda x . \suc\,x }$ has type $\Lift (\Nat_0 \to \Nat_0)$
    \end{itemize}
\end{example}

\subsubsection{Splicing}
Lastly we have \emph{splicing}. Splicing has the annotation $\spl \blank$ (``$\blank$'' is placeholder for specific term). Splice also has the highest precedence, even higher than function application. So $\spl\, f \,x$ parses to $(\spl\,f)\,x$. Splicing comes with the inference rule:

\begin{mathparpagebreakable}
\inferrule*[lab=splice]
             {\Gamma \vdash_{1,j} t : \Lift A}
             {\Gamma \vdash_{0,j} \spl{t} : A}
\end{mathparpagebreakable}

Splice is used to bring a stage 0 term that has been lifted to stage 1 back down to 0.

\begin{remark}
    We can quote any term $t$ of type $A$ if $A$ is a type in stage 0.
    However, not every term $s$ of type $B$ in stage 1 can be spliced.
    For example, $\zero_1$ cannot be spliced because $\zero_1$ is of type $\Nat_1$, and $\Nat_1$ is not a lifted type. 
\end{remark}

\subsubsection{Interaction Between Quoting and Splicing}
There are two more inference rules that tells us $\qut{\blank}$ and $\spl \blank$ are inverses of each other:
\begin{mathparpagebreakable}
    \inferrule*[lab=quote-splice]
             {\Gamma \vdash_{1,j} t : \Lift A}
             {\Gamma \vdash_{1,j} \qut{\spl{t}} = t : \Lift A}

    \inferrule*[lab=splice-quote]
             {\Gamma \vdash_{0,j} t : A}
             {\Gamma \vdash_{0,j} \spl{\qut{t}} = t : A}
\end{mathparpagebreakable}

\begin{example}[Elimination]
    The term $\suc\,(\suc\,\zero_0)$ is judgmentally the same $\Nat_0$ as $\spl \qut{\suc\,(\suc\,\zero_0)}$
\end{example}

This makes intuitive sense. If no computation is done in between lifting and splicing, we should expect to get back the same term. 

% An interesting observation is the similarity between these two inference rules and $\eta$-rules for other types:

% \begin{mathparpagebreakable}
%     \inferrule*[lab=$\Sigma$-$\eta$ (2LTT)]
%              {\Gamma \vdash_{i,j} t : \Sigma\,A\,B}
%              {\Gamma \vdash_{i,j} (\fst\,t,\,\snd\,t) = t : \Sigma\,A\,B}
             
%     \inferrule*[lab=$\Sigma$-$\eta$ (The Little Typer)]
%             {\Gamma \vdash p_1 \equiv p_2 : (\Sigma\,([x \,A])\,D)}
%             {\Gamma \vdash p_1 \equiv (\mathsf{cons}\,(\mathsf{car}\,p_2)\,(\mathsf{cdr}\,p_2)) : (\Sigma\,([x\, A])\, D)}

%     \inferrule*[lab=FunSame-$\eta$ (The Little Typer)]
%             {x \notin dom(\Gamma) \\ \Gamma \vdash f_1 \equiv f_2 : (\Pi\,([x \,Arg])\, R)}
%             {\Gamma \vdash f_1 \equiv (\lambda\,(x)\,(f_2 \,x)) : (\Pi\,([x \,Arg])\, R)}
% \end{mathparpagebreakable}

% We thus suggests that splicing then quoting ($\qut{\spl \blank}$), and quoting then splicing ($\spl \qut{\blank}$) are the $\eta$-rules for staging annotation.

One may ask why we have two annotations $\qut{\blank}$ and $\spl$ for terms, which are inverses of each other, but only one annotation, the $\Lift$ for types? To answer this, we consider the origin of 2LTT where it was intended to make meta-theoretical results of homotopy type theory internal \cite{2ltt}. We now understand stage 0 being the object-level and stage 1 being the meta-level. There is usually a way of representing object or concepts of the object-level in meta-level. However, with object-level usually being simpler, there are concepts in the meta-level which cannot be represented in the object-level. To make an analogy, $\mathbb N$ in meta-level might be defined as a subset of $\mathbb R$, while $\mathbb N$ in object-level is defined using Peano Arithmetic. In such situation, we cannot guarantee the correctness of bringing $\mathbb N$ from meta-level down to object level. 

In fact, $\qut{\spl \blank}$ is not the identity function on all terms $t$ in stage 1, but only terms like $\qut{s}$ that were lifted from stage 0. The inventors of 2LTT phrased it as such:
\begin{quote}
    One intuition for the two levels is as follows: from a type in HoTT [Homotopy Type Theory, the inner type theory], we can extract a statement that can be phrased in the meta-theory. From a meta-theoretical statement \emph{about} HoTT, it is not always possible to construct a type. Thus, we can convert inner types into outer one, but not always vice versa \cite{2ltt}.
\end{quote}