\documentclass[handout]{beamer}
% \documentclass{beamer}
\usepackage{xcolor}
\usepackage{amsmath}
\usepackage{amssymb}
\usepackage{multirow,longtable,booktabs,array}
\usepackage{amsfonts}
\usepackage{listings}
\usepackage{fontspec}
\usepackage{graphicx}
\usepackage{hyperref}
\usepackage{float}
\usepackage{tikz}
\usepackage{fancybox} % https://tex.stackexchange.com/a/85492
\usepackage{mathpartir}

\usepackage[style=alphabetic,sorting=ynt,backend=bibtex]{biblatex}
\addbibresource{refs.bib}


\newcommand\rarr\rightarrow
\newcommand\gray{\textcolor{gray}}

\setmonofont{iosevka-custom-regular.ttf}
\setmainfont{Roboto-Regular.ttf}

\hypersetup{
    colorlinks=true,
    linkcolor=blue,
    urlcolor=cyan,
}


\newcommand{\Lift}{{\Uparrow}}
\newcommand{\spl}{{\sim}}
\newcommand{\qut}[1]{\langle #1\rangle}

\renewcommand{\mit}[1]{\mathit{#1}}
\newcommand{\msf}[1]{\mathsf{#1}}
\newcommand{\mbb}[1]{\mathbb{#1}}
\newcommand{\mbf}[1]{\mathbf{#1}}
\newcommand{\bs}[1]{\boldsymbol{#1}}
\newcommand{\wh}[1]{\widehat{#1}}
\newcommand{\ext}{\triangleright}
\newcommand{\Code}{\msf{Code}}
\newcommand{\El}{\msf{El}}
\newcommand{\lam}{\msf{lam}}
\newcommand{\app}{\msf{app}}
\newcommand{\NatElim}{\msf{NatElim}}
\newcommand{\y}{\msf{y}}

\newcommand{\U}{\msf{U}}
\newcommand{\Con}{\msf{Con}}
\newcommand{\Sub}{\msf{Sub}}
\newcommand{\Ty}{\msf{Ty}}
\newcommand{\Tm}{\msf{Tm}}

\newcommand{\Set}{\mathsf{Set}}
\newcommand{\Prop}{\mathsf{Prop}}
\newcommand{\Rep}{\msf{Rep}}
\newcommand{\blank}{{\mathord{\hspace{1pt}\text{--}\hspace{1pt}}}}
\newcommand{\emb}[1]{\ulcorner#1\urcorner}


\newcommand{\refl}{\msf{refl}}
\newcommand{\Bool}{\msf{Bool}}
\newcommand{\true}{\msf{true}}
\newcommand{\false}{\msf{false}}
\newcommand{\True}{\msf{True}}
\newcommand{\False}{\msf{False}}
\newcommand{\List}{\msf{List}}
\newcommand{\nil}{\msf{nil}}
\newcommand{\cons}{\msf{cons}}
\newcommand{\Nat}{\msf{Nat}}
\newcommand{\zero}{\msf{zero}}
\newcommand{\suc}{\msf{suc}}
\renewcommand{\tt}{\msf{tt}}
\newcommand{\fst}{\msf{fst}}
\newcommand{\snd}{\msf{snd}}
\newcommand{\mylet}{\msf{let}}
\newcommand{\emptycon}{\scaleobj{.75}\bullet}
\newcommand{\id}{\msf{id}}

\newcommand{\p}{\mathsf{p}}
\newcommand{\q}{\mathsf{q}}


\theoremstyle{plain}
% \newtheorem{theorem}{Theorem}[section]

\theoremstyle{definition}
% \newtheorem{definition}{Definition}[section]
% \newtheorem{example}{Example}[subsection]
\newtheorem{property}{Property}[section]
% \newtheorem{corollary}{Corollary}[theorem]
% \newtheorem{lemma}[theorem]{Lemma}

\theoremstyle{remark}
\newtheorem{notation}{Notation}
\newtheorem{remark}{Remark}


\addbibresource{refs.bib}

\makeatletter
\newenvironment{CenteredBox}{% 
\begin{Sbox}}{% Save the content in a box
\end{Sbox}\centerline{\parbox{\wd\@Sbox}{\TheSbox}}}% And output it centered
\makeatother

\lstset{
    breakatwhitespace=false,         
    breaklines=true,                 
    captionpos=b,                    
    keepspaces=true,                 
    % numbers=left,
    inputencoding=utf8,
    % extendedchars=true,
    numbersep=5pt,                  
    showspaces=false,                
    showstringspaces=false,
    showtabs=false,                  
    tabsize=4,
    basicstyle=\ttfamily\footnotesize\color[HTML]{000000},
    backgroundcolor=\color[HTML]{ffffff},   
    numberstyle=\color[HTML]{333333},
    sensitive=true,
    columns=fullflexible,
	keepspaces,
	showstringspaces=false,
}

\newcommand\blue{\color[HTML]{3eabc9}}
\newcommand\lime{\color[HTML]{20b027}}

\lstdefinelanguage{pie} {
	morecomment=[l]{;},
	commentstyle={\color[HTML]{7f7f7f}\slshape},
    literate=%
        {lambda}{{\color[HTML]{20b027}λ}}1
        {Sigma}{{\color[HTML]{3eabc9}Σ}}1
        {TODO}{{\color[HTML]{ff0000}TODO}}4
        {\\ne}{≠}1
        {\\bot}{⊥}1
        {\\neg}{¬}1
        {_1}{₁}1
        {\\forall}{∀}1
        {!=}{!=}2
        {\\rarr}{→}1
        {_n}{ₙ}1
        {Pi}{{\color[HTML]{3eabc9}Π}}1,
    % type constructors
	keywords={
        Pi,->,Nat,U,Even,Sigma,Odd,
        Pair,=,Absurd,Trivial,Either,
        Dec,Vec
    },
    keywordstyle=\blue,
    % value constructors
	keywords={[2]
        lambda,+,*,add1,the,cons,same,car,
        cong,cdr,iter-Nat,replace,rec-Nat,
        which-Nat,left,right,ind-Vec,vec::,
        =consequence, sub1=, use-Nat=, zero-not-add1
    },
    keywordstyle={[2]\lime},
    % special
    keywords={[3]
        zero, nil, vecnil, 0, define, claim,
    },
    keywordstyle={[3]\color[HTML]{0000ff}},
	alsoletter={',-,>,:,=,+,*},
}


\setbeamercolor{frame number in head/foot}{fg=black}
\setbeamerfont{frame number in head/foot}{size=\small}
% \setbeameroption{show notes on second screen=right}
\setbeamertemplate{footline}[frame number]
\lstset{language=pie}

\newenvironment{fr}[1][]
{\begin{frame}[fragile,environment=fr,#1]}
{\end{frame}}

\title{Staged Compilation with Two-Level Type Theory}
\date{2023-08-11}

% \AtBeginSection[]{
%   \begin{frame}
%     \frametitle{Table of Contents}
%     \tableofcontents[currentsection]
%   \end{frame}
% }

\begin{document}

\frame{\titlepage}
% \begin{frame}
% \frametitle{Table of Contents}
% \tableofcontents
% \end{frame}

\section{Staged Compilation}
\begin{frame}{\secname}
Staged compilation separates code compilation in at two stages.
\begin{itemize}
    \item \pause Compile time (aka meta-level, indexed 1) stage and runtime (aka object-level, indexed 0) stage.
    \item \pause Each stage can have their own language.
\end{itemize}
\pause 
A staging algorithm converts/stages a metaprogram to a program with only runtime language.  
\begin{itemize}
    \item\pause A metaprogram is a term with runtime type but uses type/terms from the compile-time language through staging annotations.
\end{itemize}
\pause
Examples where staged compilation is useful:
    \begin{itemize}
        \item\pause  Metaprogramming: evaluate the code-generation annotations (e.g. macros, inlining) to runtime language, i.e. without annotations.
        \item\pause  Domain specific languages: e.g. \href{https://learn.microsoft.com/en-us/dotnet/csharp/programming-guide/concepts/linq/standard-query-operators-overview}{LINQ} (C\#)
    \end{itemize}
    \note{
    \begin{itemize}
        \item haskell deriving, rust \#[derives] in conclusion
    \end{itemize}
    }
\end{frame}

\subsection{Two-Level Type Theory}
\begin{fr}{\subsecname}

A staging algorithm needs to be sound:

\begin{itemize}
    \item\pause Any well-typed metaprogram should be staged into a well-typed runtime code.
    \item\pause The resulting code of staging does not use any types/terms from the compile-time language.
\end{itemize}
\pause 
To justify the the soundness of staging

\begin{itemize}
    \item\pause  Two-level type theory can be applied.
    \item\pause  Treat the two stages as separate type systems.
    \item\pause  Restrict the interaction between stages with explicit annotations.
\end{itemize}
\pause
Unlike previous works like MetaML \cite{metaml}, staged compilation with 2LTT supports dependent types.

\note{trade offer}
\end{fr}

\section{Notation}

\subsection{$\Pi$-Types}
\begin{fr}{\subsecname}

To adhere to the notations from the original paper on staged compilation \cite{staged}, we introduce new notations in comparison with Pie.

\begin{itemize}
    \item \pause Dependent function types ($\Pi$-types)

    $$f:(a:A)\to (b:B)\to C$$
\begin{CenteredBox}
\lstinline{(claim f (Pi ([a A] [b B]) C))}
\end{CenteredBox}
    \item \pause Functions (lambdas)
    $$f:=\lambda\,a\,b.\,\mathsf{body}$$
\begin{CenteredBox}
\lstinline{(define f (lambda (a b) body))}
\end{CenteredBox}
\end{itemize}
\end{fr}

%==============================================================================


\section{Moving Between Stages}
\begin{fr}{\secname}
    We have two universes of types, one for each stage:
    \begin{itemize}
        \item $\U_0$ for the universe of stage 0 (recall 0 is index for runtime/object stage)
        \item $\U_1$ for the universe of stage 1 (recall 1 is index for compile-time/meta stage)
    \end{itemize}
    For interaction between stages, we define three staging annotations on the compile-time level.
\end{fr}

\subsection{Lifting: Compute Runtime Expressions at Compile-Time}

\begin{frame}{\subsecname}
    Given $A : \U_0$, we have $\Lift A : \U_1$
    \begin{itemize}
        \item The lifted type $\Lift A:\U_1$ is the type of metaprograms that compute runtime expression of type $A:\U_0$.
    \end{itemize}
\end{frame}

\subsection{Quoting: Metaprograms from Runtime Terms}
\begin{fr}{\subsecname}
Given $t:A$ and $A:\U_0$, we have $\qut{t}:\Lift A$ and $\Lift A: \U_1$
\begin{itemize}
    \item The quoted term $\qut{t}:\Lift A$ is a metaprogram that immediately yields the term $t:A$.
\end{itemize}
\end{fr}

\subsection{Splicing: Executing Metaprograms During Staging}
\begin{fr}{\subsecname}
Given $t:\Lift A$ and $\Lift A: \U_1$, we have $\spl t: A$ and $ A: \U_0$
\begin{itemize}
    \item The spliced metaprogram $\spl t$ will be executed during staging, and substituted by result expression.
\end{itemize}

\end{fr}

\subsection{Moving Between Stages}
\begin{fr}{\subsecname}
    With the three staging annotations for moving between stages:
    \begin{itemize}
        \item \emph{Lifting}: Given $A : \U_0$, we have $\Lift A : \U_1$
        \item \emph{Quoting}: Given $t:A$ and $A:\U_0$, we have $\qut{t}:\Lift A$
        \item \emph{Splicing}: Given $t:\Lift A$ and $\Lift A: \U_1$, we have $\spl t: A$
    \end{itemize}
    We have two equalities:
    $$\begin{aligned}
        \spl \qut{t} &= t\\
        \qut{\spl s} &= s
    \end{aligned}$$
\end{fr}

%==============================================================================

\section{The Natural Eliminator}
\begin{fr}{\secname}

We introduce natural numbers for each stage $i\in\{0,1\}$

\begin{align*}
    \pause\mathsf{Nat}_i&:U_i\\
    \pause\mathsf{zero}_i&:\mathsf{Nat}_i\\
    \pause\mathsf{suc}_i&:\mathsf{Nat}_i\to\mathsf{Nat}_i\\
    \pause\mathsf{NatElim}_i&:(P:\mathsf{Nat}_i\to\mathsf{U}_{i,j})\\
    &\to P\,\mathsf{zero}_i\\
    &\to ((n:\mathsf{Nat}_i)\to P\,n\to P\,(\mathsf{suc}_i\,n))\\
    &\to(t:\mathsf{Nat}_i)\\
    &\to P\,t\\
\end{align*}

\pause

For simplicity, let us define $\mathsf{iter}_i$ to represent \texttt{iter-Nat} from Pie

\begin{align*}
    \mathsf{iter}_i&:(X:U_i)\to\mathsf{Nat}_i\to X\to (X\to X)\to X\\
    \mathsf{iter}_i&:=\lambda\,X\,t\,z\,s.\,\mathsf{NatElim}_i\,(\lambda\,\gray n.\,X)\,z\,(\lambda\,\gray n\,\mathsf{acc}.\,s\,\mathsf{acc})\,t\\
\end{align*}

\end{fr}

\subsection{Addition and Multiplication}
\begin{fr}{\subsecname}
$$
\begin{aligned}
    \mathsf{iter}_i&:(X:U_i)\to\mathsf{Nat}_i\to X\to (X\to X)\to X\\
    \mathsf{iter}_i&:=\lambda\,X\,t\,z\,s.\,\mathsf{NatElim}_i\,(\lambda\,\gray n.\,X)\,z\,(\lambda\,\gray n\,\mathsf{acc}.\,s\,\mathsf{acc})\,t\\
\end{aligned}
$$
\pause
Similar to the definition of \texttt+ and \texttt* in Pie, we can implement them in 2LTT as well.

\begin{itemize}
    \item\pause We put $\mathsf{add}_0$ to stage $0$:
    $$
    \begin{aligned}
        \mathsf{add}_0&:\mathsf{Nat}_0\to\mathsf{Nat}_0\to\mathsf{Nat}_0\\
        \mathsf{add}_0&:=\lambda\,a\,b.\,\mathsf{iter}_0\,\mathsf{Nat}_0\,a\,b\,(\lambda\,n.\mathsf{suc}_0\,n)\\
    \end{aligned}
    $$
    \item \pause For multiplication, it takes a compile-time number $x:\mathsf{Nat}_1$ and produces a metaprogram that computes the product with $x$ at runtime.
    $$
    \begin{aligned}
        \pause\mathsf{mul}_1&:\mathsf{Nat}_1\to{\Uparrow}\mathsf{Nat}_0\to{\Uparrow}\mathsf{Nat}_0\\
        \pause\mathsf{mul}_1&=\lambda\,x\,t.\,\mathsf{iter}_1\,({\Uparrow}\mathsf{Nat}_0)\,x\,\langle\mathsf{zero}_0\rangle\,\langle \mathsf{add}_0\,{\sim}t\rangle\\
    \end{aligned}
    $$
\end{itemize}
\end{fr}

\subsection{The Staging Process}
\begin{fr}{\subsecname}
$$
\begin{aligned}
\mathsf{iter}_i&:(X:U_i)\to\mathsf{Nat}_i\to X\to (X\to X)\to X\\
\mathsf{iter}_i&:=\lambda\,X\,t\,z\,s.\,\mathsf{NatElim}_i\,(\lambda\,\gray n.\,X)\,z\,(\lambda\,\gray n\,\mathsf{acc}.\,s\,\mathsf{acc})\,t\\
\mathsf{add}_0&:\mathsf{Nat}_0\to\mathsf{Nat}_0\to\mathsf{Nat}_0\\
\mathsf{add}_0&:=\lambda\,a\,b.\,\mathsf{iter}_0\,\mathsf{Nat}_0\,a\,b\,(\lambda\,n.\mathsf{suc}_0\,n)\\
\mathsf{mul}_1&:\mathsf{Nat}_1\to{\Uparrow}\mathsf{Nat}_0\to{\Uparrow}\mathsf{Nat}_0\\
\mathsf{mul}_1&=\lambda\,x\,t.\,\mathsf{iter}_1\,({\Uparrow}\mathsf{Nat}_0)\,x\,\langle\mathsf{zero}_0\rangle\,\langle \mathsf{add}_0\,{\sim}t\rangle\\
\end{aligned}
$$


\pause With $\mathsf{add}_0$ being a function on stage $0$ and $\mathsf{mul}_1$ on stage $1$, a metaprogram, for instance\pause
$$
\begin{aligned}
    \mathsf{double}&:\mathsf{Nat}_0\to\mathsf{Nat}_0\\
    \mathsf{double}&:=\lambda\,x.\,{\sim}(\mathsf{mul}_1\,2\,\langle x\rangle)\\
\end{aligned}
$$

will get staged to \pause 
$$
\begin{aligned}
    \mathsf{double}&:=\lambda\,x.\,\mathsf{add}_0\,x\,(\mathsf{add}_0\,x\,\mathsf{zero}_0)
\end{aligned}
$$
\end{fr}



%==============================================================================

\section{Formal Inference Rules}
\begin{fr}{\secname}
    \begin{mathparpagebreakable}
        \inferrule*[lab=lift]
                 {\Gamma \vdash_{0,j} A}
                 {\Gamma \vdash_{1,j} \Lift A}
    
        \inferrule*[lab=quote]
                 {\Gamma \vdash_{0,j} t : A}
                 {\Gamma \vdash_{1,j} \qut{t} : \Lift A}
    
        \inferrule*[lab=splice]
                 {\Gamma \vdash_{1,j} t : \Lift A}
                 {\Gamma \vdash_{0,j} \spl{t} : A}

        \inferrule*[lab=quote-splice]
             {\Gamma \vdash_{1,j} t : \Lift A}
             {\Gamma \vdash_{1,j} \qut{\spl{t}} = t : \Lift A}

    \inferrule*[lab=splice-quote]
             {\Gamma \vdash_{0,j} t : A}
             {\Gamma \vdash_{0,j} \spl{\qut{t}} = t : A}
    \end{mathparpagebreakable}
\end{fr}

%==============================================================================

\section{Limitation of Staging}
\begin{fr}{\secname}
The original purpose of 2LTT is to express meta-theoretical statements about homotopy type theory (HoTT)
    \begin{itemize}
        \item\pause ``From a type in HoTT, we can extract a statement that can be phrased in the meta-theory. From a meta-theoretical statement \emph{about} HoTT, it is not always possible to construct a type. Thus, we can convert inner types into outer one, but not always vice versa.'' \cite{2ltt}
        \item\pause Therefore cannot splice arbitrary stage 1 term.
        \item\pause Stage 0 don't always have ways to represent types in stage 1.
        \item\pause $\spl \zero_1$ would be invalid.
    \end{itemize}
\end{fr}

%==============================================================================

\section{Isomorphism Between Types}
\begin{fr}{\secname}
    \begin{align*}
    \Lift ((a: A) \to B\,a) &\simeq (a: \Lift A) \to \Lift (B\,\spl a) \\ \\
    \Lift ((a:A) \times B\,a) &\simeq ((a: \Lift A) \times \Lift (B\,\spl a)) \\
    \end{align*}
\end{fr}

%==============================================================================

\subsection{Isomorphism Example}
\begin{fr}{\subsecname}
    \centering
    \begin{align*}
      & \hspace{-10em}\rlap{$\mit{pres}_\to : \Lift((x : A) \to B\,x) \to ((x : \Lift A) \to \Lift\,(B\,\spl x))$}\\
      & \hspace{-10em}\mit{pres}_\to\,f := \lambda\,x.\,\,\qut{\spl f\,\spl x}\\
      & \hspace{-10em}\mit{pres}_\to := \lambda\,f\,x.\,\,\qut{\spl f\,\spl x}\\
    \\
      & \hspace{-10em}\rlap{$\mit{pres}_\to^{-1} : ((x : \Lift A) \to \Lift\,(B\,\spl x)) \to \Lift((x : A) \to B\,x)$}\\
      & \hspace{-10em}\mit{pres}_\to^{-1}\,f:= \qut{\lambda\,x.\,\,\spl(f\,\qut{x})} \\
      & \hspace{-10em}\mit{pres}_\to^{-1}:= \lambda\,f.\,\,\qut{\lambda\,x.\,\,\spl(f\,\qut{x})}
    \end{align*}
\end{fr}

%==============================================================================

\begin{fr}{\subsecname}
    $$
    \begin{aligned}
      \mit{pres}_\to (\mit{pres}_\to^{-1} f)&=\,\,\,(\lambda\,f\,x.\,\,\qut{\spl f\,\spl x})\,\, ((\lambda\,f.\,\,\qut{\lambda\,x.\,\,\spl(f\,\qut{x})})\,\, f) \\
      \pause&=_\beta (\lambda\,f\,x.\,\,\qut{\spl f\,\spl x})\,\, \qut{\lambda\,x.\,\,\spl(f\,\qut{x})}\\
      \pause&=_\beta \lambda\,x.\,\,\qut{\spl \qut{\lambda\,x.\,\,\spl(f\,\qut{x})}\,\spl x} \\
      \pause&=\,\,\, \lambda\,x.\,\,\qut{(\lambda\,x.\,\,\spl(f\,\qut{x}))\,\spl x} \\
      \pause&=_\beta \lambda\,x.\,\,\qut{\spl(f\,\qut{\spl x})} \\
      \pause&=\,\,\, \lambda\,x.\,\,\qut{\spl(f\, x)} \\
      \pause&=\,\,\, \lambda\,x.\,\,(f\, x) \\
      \pause&=_\eta f
    \end{aligned}
    $$
\end{fr}

%==============================================================================

% \section{Limit}
% \begin{fr}{\secname}
%     \begin{itemize}
%         \item ``The lack of elimination principles for $\Lift \blank$ means that we cannot inspect the internal structure of expressions''
%         \item Cannot find useful mapping $\qut{\Bool_0} \to \Bool_1$
%     \end{itemize}
% \end{fr}

%==============================================================================

\section{Result of 2LTT}
\begin{fr}{\secname}
    \begin{itemize}
        \item Staging: Given a program $t : A$, where $A : \U_0$. \emph{Staging} computes metaprograms and replace all splices in $t$ and $A$ with resulting runtime expression.
        \item 2LTT guarantees resulting computation does not contain more splices.
        \item Regardless of the body, if you have a runtime type program, it can be turned into a program using strictly stage 0 terms and constructors.
    \end{itemize}
\end{fr}

%==============================================================================

\section{Bibliography}
\begin{fr}{\secname}
    \printbibliography[heading=none]
\end{fr}

\begin{frame}{}
  \centering \Large
  \emph{End}
\end{frame}
\end{document}